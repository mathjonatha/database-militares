\begin{question}%[codigo:EAM202301AMX; concurso:EAM; ano:2023; assunto:; alternativa:E]
O chão da Sala de Estado da EAMCE é retangular e suas dimensões são 3,52m e 4,16m. Esse chão será revestido com pisos quadrados, de dimensões iguais, inteiros, de forma que não fiquem espaços vazios entre pisos vizinhos. Os pisos serão escolhidos de modo que tenham a maior dimensão possível. Com base nessa situação, assinale a opção que apresenta o intervalo que contém a medida do lado do piso ideal.
    \begin{tasks}
        \task Menos de 15 cm.
        \task Mais de 15 cm e menos de 20 cm.
        \task Mais de 20 cm e menos de 25 cm.
        \task Mais de 25 cm e menos de 30 cm.
        \task Mais de 30 cm.
    \end{tasks}
\end{question}

\begin{question}%[codigo:EAM202302AMX; concurso:EAM; ano:2023; assunto:; alternativa:B]
Migulito e Ditão são dois integrantes de um seleto grupo de vinte tricolores. Será formada uma comissão de cinco pessoas entre seus membros para organizar a feste de confraternização do grupo. Com base nessas informações, assinale a opção que indica de quantas maneiras distintas essa comissão poderá ser formada, de modo que apenas um deles esteja presente.
    \begin{tasks}
        \task 9 180.
        \task 6 120.
        \task 5 400.
        \task 4 590.
        \task 3 060.
    \end{tasks}
\end{question}

\begin{question}%[codigo:EAM202303AMX; concurso:EAM; ano:2023; assunto:; alternativa:E]
Em um exercício da Marinha do Brasil, cinco navios estavam posicionados nos vértices de um pentágono regular imaginário. Assinale a opção que indica a maior distância entre dois desses navios, sabendo que a menor distância entre dois navios mais próximos é 100 milhas marítimas. Dados: \(\sqrt{2} = 1,41\) ; \( \sqrt{3} = 1,73\) e \(\sqrt{5} = 2,24\).
    \begin{tasks}
        \task 122 milhas marítimas.
        \task 132 milhas marítimas.
        \task 142 milhas marítimas.
        \task 152 milhas marítimas.
        \task 162 milhas marítimas.
    \end{tasks}
\end{question}

\begin{question}%[codigo:EAM202304AMX; concurso:EAM; ano:2023; assunto:; alternativa:C]
Durante um exercício naval, a Fragata Constituição lançou um míssil antinavio de superfície (MANSUP), cuja trajetória foi determinada pela parábola de equação cartesiana \(y=-x^2 + 20x\), na qual \(y\) representa a altura do míssil e \(x\), o tempo ocorrido após o lançamento. Do mesmo ponto de lançamento do MANSUP, outro míssil lançado, a fim de interceptá-lo no ponto mais alto da sua trajetória. Sabendo que a trajetória do segundo míssil foi retilínea, assinale a opção que apresenta a equação cartesiana desa trajetória.
    \begin{tasks}
        \task \(y = x\)
        \task \(y = 5x\)
        \task \(y = 10x\)
        \task \(y = 15x\)
        \task \(y = 20x\)
    \end{tasks}
\end{question}

\begin{question}%[codigo:EAM202305AMX; concurso:EAM; ano:2023; assunto:; alternativa:]
A taxa de crescimento da população de uma colônia de bactérias é de 2\% ao mês. Assinale a opção que indica o intervalo de tempo em que o número de bactérias dessa colônia dobra: Dados: \(\log 0,2 = 0,70\), \(\log 2 = 0,30\), \(\log 1,2 = 0,08\) e \(\log 1,02 = 0,008\).

    \begin{tasks}
        \task Durante o 35\textordmasculine~ mês após o início da observação.
        \task Durante o 36\textordmasculine~ mês após o início da observação.
        \task Durante o 37\textordmasculine~ mês após o início da observação.
        \task Durante o 38\textordmasculine~ mês após o início da observação.
        \task Durante o 39\textordmasculine~ mês após o início da observação.
    \end{tasks}
\end{question}

\begin{question}%[codigo:EAM202306AMX; concurso:EAM; ano:2023; assunto:; alternativa:D]
Pelo ponto médio da diagonal de um cubo de aresta 2 cm foi traçado um plano perpendicular a essa diagonal. Assinale a opção que apresenta a área da figura plana obtida pela interseção desse plano com as faces do cubo.
    \begin{tasks}
        \task \(\sqrt{3}\) cm\(^2\).
        \task \(2\sqrt{3}\) cm\(^2\).
        \task \(3\sqrt{3}\) cm\(^2\).
        \task \(4\sqrt{3}\) cm\(^2\).
        \task \(5\sqrt{3}\) cm\(^2\).
    \end{tasks}
\end{question}

\begin{question}%[codigo:EAM202307AMX; concurso:EAM; ano:2023; assunto:; alternativa:]
Um quadrilátero de vértices consecutivos \(A_1,A_2,A_3,A_4\), tem as distâncias entre os vértices \(A_1\) e \(A_3\) indicadas na matriz

\[
A = \begin{bmatrix}
0   & 4,5 & x & 4,5\\
4,5 & 0   & 3 & 3  \\
y   & 3   & 0 & 3  \\
4,5 & 3   & 3 & 0
\end{bmatrix}
\]

pelo elemento \(a_{ij}\), com valores em \(cm\) e \(x,y \in \mathbb{R}\). Determine a área desse quadrilátero e assinale a opção correta.
    \begin{tasks}
        \task \(\frac{9}{2} (2\sqrt{2} + \sqrt{3})\) cm\(^2\)
        \task \(\frac{9}{4} (2\sqrt{2} + \sqrt{3})\) cm\(^2\)
        \task \(\frac{9}{2} (\sqrt{2} + \sqrt{3})\) cm\(^2\)
        \task \(\frac{9}{4} (\sqrt{2} + \sqrt{3})\) cm\(^2\)
        \task \(\frac{9}{2} (2\sqrt{2} - \sqrt{3})\) cm\(^2\)
    \end{tasks}
\end{question}

\begin{question}%[codigo:EAM202308AMX; concurso:EAM; ano:2023; assunto:; alternativa:B]
Os instrutores Adilson (1), Daverson (2), Estácio (3), Isnard (4) e Vicente (5) foram convocados para elaborar o CPAEAM2023. Eles votaram entre si e elegeram o presidente dessa banca. Seus votos foram organizados segundo a matriz \(P\) abaixo, em que cada elemento \(p_{ij}\) é igual a \(1\)(um), se \(i\) votou em \(i\), e \(0\), se \(i\) não votou em \(i\).

\[
P = \begin{bmatrix}
0 & 1 & 0 & 1 & 0\\
1 & 1 & 0 & 0 & 0\\
0 & 0 & 1 & 0 & 1\\
0 & 0 & 1 & 0 & 0\\
1 & 1 & 0 & 1 & 0
\end{bmatrix}
\]

O número de votos foi livre e cada um deles pode votar em si mesmo. Assinale a opção que apresenta quem foi escolhido como presidente da banca.
    \begin{tasks}
        \task Adilson
        \task Daverson
        \task Estácio
        \task Isnard
        \task Vicente
    \end{tasks}
\end{question}

\begin{question}%[codigo:EAM202309AMX; concurso:EAM; ano:2023; assunto:; alternativa:A]
Na recepção da passagem de comando da EAMSC os drinks foram servidos em taças cônicas de 320 ml. Um dos convidados que estava com a taça completamente cheia, resolveu beber a quantidade do drink suficiente para que a bebida restante ficasse na metade da altura da taça, sem considerar sua base. Com base nessas informações, assinale a opção que apresenta a quantidade de bebida que ele sorveu nesse gole.
    \begin{tasks}
        \task 280 ml.
        \task 200 ml.
        \task 160 ml.
        \task 80 ml.
        \task 40 ml.
    \end{tasks}
\end{question}

\begin{question}%[codigo:EAM202310AMX; concurso:EAM; ano:2023; assunto:; alternativa:C]
Considere as equações \(x^2 - 9y^2 - 6x - 18y - 9 = 0\), \(x^2 + y^2 - 2x + 4y + 1 = 0\) e \(x^2 -4x - 4y + 8 = 0\), com \((x,y) \in \mathbb{R}^2\). Analise e assinale a opção que apresenta, respectivamente, as representações geométricas das equações.
    \begin{tasks}
        \task Hipérbole, elipse, parábola.
        \task Hipérbole, circunferência, reta.
        \task Hipérbole, circunferência, parábola.
        \task Elipse, circunferência, parábola.
        \task Elipse, circunferência, reta.
    \end{tasks}
\end{question}

\begin{question}%[codigo:EAM202311AMX; concurso:EAM; ano:2023; assunto:; alternativa:C]
Sejam \(a\) e \(b\) as soluções reais da equação
\(\frac{4 + \sqrt{2x^2 - 7}}{\sqrt{2x^2-7}} = \sqrt{2x^2 - 7} + 4\), com \(a > b\).
\\
Assinale a opção que apresenta o valor correto para \(a^b\).
    \begin{tasks}
        \task \(-4\).
        \task \(-0,25 \).
        \task \( 0,25\).
        \task \( 1\).
        \task \( 4\).
    \end{tasks}
\end{question}

\begin{question}%[codigo:EAM202312AMX; concurso:EAM; ano:2023; assunto:; alternativa:E]
O polinômio \(P(x) = -x^3 + 2x^2 + mx + n\) é divisível simultaneamente pelos polinômios \(Q(x) = x - 1\) e \(R(x) = x-2\). Determine o valor de \(m-n\) e assinale a opção correta.
    \begin{tasks}
        \task \(-1 \).
        \task \( 0\).
        \task \( 1\).
        \task \( 2\).
        \task \( 3\).
    \end{tasks}
\end{question}

\begin{question}%[codigo:EAM202301AMX; concurso:EAM; ano:2023; assunto:; alternativa:]
Através de um ponto \(P\) qualquer, tomado dentro de um triângulo, são traçadas três retas paralelas aos lados desse triângulo. Essas retas dividem a superfície do triângulo em seis partes, três das quais são triângulos de áreas \(4\) cm\(^2\), \(9\) cm\(^2\) e \(16\) cm\(^2\). Assinale a opção que apresenta a àrea do triângulo original.
    \begin{tasks}
        \task \(64\) cm\(^2\).
        \task \(72\) cm\(^2\).
        \task \(81\) cm\(^2\).
        \task \(90\) cm\(^2\).
        \task \(100\) cm\(^2\).
    \end{tasks}
\end{question}

\begin{question}%[codigo:EAM202301AMX; concurso:EAM; ano:2023; assunto:; alternativa:]
Um concurso para as Escolas de Aprendizes-Marinheiros ofereceu uma certa quantidade de vagas, das quais \(1/5\) foi destinado para a área de Eletroeletrônica. O restante foi dividido igualmente entre as áreas Profissional de Apoio e Mecânica. As áreas de Mecânica e Eletroeletrônica contam com uma subespecialidade em comum, chamada Armamento de Aviação, que recebeu 30\% das vagas de Mecânica e 50\% das vagas de Eletroeletrônica. Em relação ao concurso em questão, determine o percentual de vagas destinadas a Armamento de Aviação e assinale a opção correta.
    \begin{tasks}
        \task 22\%.
        \task 26\%.
        \task 30\%.
        \task 40\%.
        \task 80\%.
    \end{tasks}
\end{question}

\begin{question}%[codigo:EAM202301AMX; concurso:EAM; ano:2023; assunto:; alternativa:]
Ao se tentar abrir uma porta com um chaveiro que contém várias chaves parecidas, há quem afirme que a porta será aberta somente na última tentativa. Pedro recebeu a tarefa de guardar equipamentos no paiol e, para tal, deram-lhe um chaveiro contento oito chaves. Assim, calcule a probabilidade de que ele acerte somente na última tentativa e assinale a opção correta.
    \begin{tasks}
        \task 12,5\%.
        \task 25,0\%.
        \task 50,0\%.
        \task 75,0\%.
        \task 87,5\%.
    \end{tasks}
\end{question}