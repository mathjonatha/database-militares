\begin{question}%[codigo:EAM201901AMX; concurso:EAM; ano:2019; assunto:; alternativa:]
Seja \(f\) uma função real, definida por \(f(x) = x^2 - 3x +2\). O conjunto imagem dessa função é o intervalo:
    \begin{tasks}
        \task \(\left[ -\frac{1}{3}; + \infty \right)\).
        \task \(\left[ -\frac{1}{6}; + \infty \right)\).
        \task \(\left[ -\frac{1}{4}; + \infty \right)\).
        \task \(\left[ -\frac{1}{2}; + \infty \right)\).
        \task \(\left[ \frac{1}{4}; + \infty \right)\).
    \end{tasks}
\end{question}

\begin{question}%[codigo:EAM201902AMX; concurso:EAM; ano:2019; assunto:; alternativa:]
A expressão \(\frac{2+a^2-3a}{6+a^2-5a} \div \frac{4+a^2-5a}{12-7a+a^2}\), quando simplificada, considerando a condição de existência dessa simplificação, tem como resultado:
    \begin{tasks}
        \task \( a^2 + 1\).
        \task \( a+1\).
        \task \( 2\).
        \task \( 1\).
        \task \( a-1\).
    \end{tasks}
\end{question}

\begin{question}%[codigo:EAM201903AMX; concurso:EAM; ano:2019; assunto:; alternativa:]
Sendo um hexágono regular inscrito em um círculo de raio 2, calcule a medida da diagonal maior desse hexágono e assinale a opção correta.
    \begin{tasks}
        \task \(4\).
        \task \(4\sqrt{3}\).
        \task \(8\).
        \task \(6\sqrt{3}\).
        \task \(12\).
    \end{tasks}
\end{question}

\begin{question}%[codigo:EAM201904AMX; concurso:EAM; ano:2019; assunto:; alternativa:]
Para vender seus produtos, um comerciante reduziu os preços dos brinquedos em 10\%. Depois que houve uma recuperação nas vendas, decidiu restaurar o valor antigo. Sendo assim, o novo preço deve ser aumentado aproximadamente em:
    \begin{tasks}
        \task 9\%.
        \task 11\%.
        \task 13\%.
        \task 15\%.
        \task 17\%.
    \end{tasks}
\end{question}

\begin{question}%[codigo:EAM201905AMX; concurso:EAM; ano:2019; assunto:; alternativa:]
O conjunto solução, nos reais, na inequação \(\frac{5}{x-1}> 1\) é o intervalo:
    \begin{tasks}
        \task \(]5,6[\).
        \task \(]-\infty,6[\).
        \task \(\mathbb{R}\).
        \task \(]1, +\infty[\).
        \task \(]1,6[\).
    \end{tasks}
\end{question}

\begin{question}%[codigo:EAM201906AMX; concurso:EAM; ano:2019; assunto:; alternativa:]
Sendo \(x\) real tal que \(\sen x = \frac{m-1}{2}\) e \(\cos x = \frac{m+1}{2}\). Determine o conjunto dos valores de \(m\) e assinale a opção correta.
    \begin{tasks}
        \task \(\{-\sqrt{2},\sqrt{2}\}\)
        \task \(\{-1,+1\}\)
        \task \(\{-2,+2\}\)
        \task \(\mathbb{R}\)
        \task \(\oslash\)
    \end{tasks}
\end{question}

\begin{question}%[codigo:EAM201907AMX; concurso:EAM; ano:2019; assunto:; alternativa:]
Os lados de um triângulo medem 30cm, 70cm e 80cm. Ao traçarmos a altura desse triângulo em relação ao maior lado, dividiremos esse lado em dois segmentos. Sendo assim, calcule o valor do menor segmento em centímetros e assinale a opção correta.
    \begin{tasks}
        \task 15.
        \task 14.
        \task 13.
        \task 12.
        \task 11.
    \end{tasks}
\end{question}

\begin{question}%[codigo:EAM201908AMX; concurso:EAM; ano:2019; assunto:; alternativa:]
Um produto custa à vista R\$ 100,00 e pode ser vendido também em 2 parcelas, sendo a primeira no ato da compra, com valor de R\$ 50,00, e a segunda, a vencer em 30 dias, com o valor de R\$ 60,00. Sendo assim, calcule a taxa mensal de juros cobrado pelo vendedor e assinale a opção correta.
    \begin{tasks}
        \task 20\%.
        \task 10\%.
        \task 8\%.
        \task 6\%.
        \task 5\%.
    \end{tasks}
\end{question}

\begin{question}%[codigo:EAM201909AMX; concurso:EAM; ano:2019; assunto:; alternativa:]
Considere o triângulo \(ABC\), isósceles, de lados \(AB=AC\). Seja o ponto \(D\), sobre o lado \(BC\), de forma que o ângulo \(BAD\) é 30\(^\circ\). Seja \(E\) o ponto sobre o lado \(AC\), tal que o ângulo \(EDC\) vale \(x\) graus. Tendo em vista que o segmento \(AD\) e \(AE\) têm as mesmas medidas, é correto afirmar que o valor da quarta parte de \(x\) é:
    \begin{tasks}
        \task 3\(^\circ\).
        \task 3\(^\circ\) 20'.
        \task 3\(^\circ\) 30'.
        \task 3\(^\circ\) 35'.
        \task 3\(^\circ\) 45'.
    \end{tasks}
\end{question}

\begin{question}%[codigo:EAM201910AMX; concurso:EAM; ano:2019; assunto:; alternativa:]
Considere o gráfico abaixo de um função real, definida por \(y= ax + b\):

INSERIR FIGURA

Com base nesse gráfico, é correto afirmar que a equação que define essa função é:
    \begin{tasks}
        \task \(4y = -4x + 16\).
        \task \(4y = -4x + 8\).
        \task \(y = -2x + 4\).
        \task \(y= 2x + 2\).
        \task \(2y = x-2\).
    \end{tasks}
\end{question}

\begin{question}%[codigo:EAM201911AMX; concurso:EAM; ano:2019; assunto:; alternativa:]
Calcule o valor de \(x\), na equação: 
\(\begin{bmatrix}
x & 1  & 1\\
3 & 1 & 1\\
1 & -3 & 1
\end{bmatrix}
=24 \) e assinale a opção correta.
    \begin{tasks}
        \task 11.
        \task 10.
        \task 9.
        \task 8.
        \task 7.
    \end{tasks}
\end{question}

\begin{question}%[codigo:EAM201912AMX; concurso:EAM; ano:2019; assunto:; alternativa:]
Sejam os conjuntos \(A=\{x \in \mathbb{R}; 1 \leq x \leq 4\}, B = \{ y \in \mathbb{R}; 3 \leq y \leq 7\}\). Considerando o conjunto \(A \times B\) (\(A\) cartesiano \(B\)), pode-se afirmar que a diagonal do polígono formado por esse conjunto é representada numericamente por:
    \begin{tasks}
        \task 2.
        \task 3.
        \task 4.
        \task 5.
        \task 6.
    \end{tasks}
\end{question}

\begin{question}%[codigo:EAM201913AMX; concurso:EAM; ano:2019; assunto:; alternativa:]
Seja \(A\) um conjunto com \(n\) elementos, tal que \(n \geq 3\). O número de subconjuntos de \(A\) com dois ou três elementos que podemos construir é igual a:
    \begin{tasks}
        \task \(\frac{(n^2-1)}{6}\).
        \task \(\frac{n-1}{6}\).
        \task \(\frac{n(n^2+1)}{6}\).
        \task \(\frac{n(n^2-1)}{6}\).
        \task \(\frac{n(n^2-1)}{5}\).
    \end{tasks}
\end{question}

\begin{question}%[codigo:EAM201914AMX; concurso:EAM; ano:2019; assunto:; alternativa:]
Observe a figura abaixo.

INSERIR FIGURA

Considerando que os triângulos \(BDA\) e \(BCA\) apresentados acima são, respectivamente, retângulos em \(D\) e \(C\), calcule o valor de \(x\) em função do lado \(c\) e assinale a opção correta.
    \begin{tasks}
        \task \(\sqrt{c^3 - 2}\).
        \task \(\sqrt{c^2 - 1}\). 
        \task \(\sqrt{c^2 + 5}\).
        \task \(\sqrt{c - 3}\).
        \task \(\sqrt{c^2 -3}\).
    \end{tasks}
\end{question}

\begin{question}%[codigo:EAM201915AMX; concurso:EAM; ano:2019; assunto:; alternativa:]
Considerando os conjuntos \(\mathbb{N},\mathbb{Z},\mathbb{Q}\) e \(\mathbb{R}\), coloque V(verdadeiro) ou F(falso) nas sentenças abaixo, assinalando a seguir a opção correta.

\begin{enumerate}[label=(~~)]
    \item \( (\mathbb{N}^{*} \cap \mathbb{Q}) = \mathbb{N}^{*}\).
    \item \( (\mathbb{Z} - \mathbb{Z}) = \mathbb{Z_{+}}\).
    \item \( ( \mathbb{R} \cup \mathbb{Z}) = \mathbb{Q}\).
\end{enumerate}
    \begin{tasks}
        \task (V)(V)(V).
        \task (V)(V)(F).
        \task (V)(F)(F).
        \task (F)(V)(F).
        \task (F)(F)(V).
    \end{tasks}
\end{question}
