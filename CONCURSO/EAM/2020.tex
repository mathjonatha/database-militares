\begin{question}%[concurso:EAM; ano:2020; assunto:; alternativa:]
Observe a figura a seguir.

INSERIR FIGURA

Nesta figura, tem-se \(\overline{AB} = \overline{AC} = 9, \overline{BC} = \overline{BD} = 6\) e ângulos \(C\hat{B}Q = Q\hat{B}D\). É correto afirmar que o cosseno do ângulo \(C\hat{B}Q\) é igual a:
    \begin{tasks}
        \task \(\frac{\sqrt{2}}{4}\).
        \task \(\sqrt{2}\).
        \task \(\frac{3}{2}\).
        \task \(\frac{\sqrt{4}}{2}\).
        \task \(\frac{2\sqrt{2}}{3}\).
    \end{tasks}
\end{question}

\begin{question}%[concurso:EAM; ano:2020; assunto:; alternativa:]
Um bar possui um alvo, como o da figura abaixo, para entretenimento dos seus clientes em lançamento de dardos. Esse alvo é formado por figuras combinadas: um semicírculo com diâmetro \(AB\), um semicírculo com diâmetro \(BC\) e um triângulo retângulo \(ABC\), conforme se observa na figura.

INSERIR FIGURA

Se o cateto \(AC\) mede 6dm, a hipotenusa \(AB\) mede 10dm e um cliente de costas para o alvo arremessa um dardo e o acerta, é correto afirmar que a probabilidade de que o dardo tenha acertado a parte sombreada do alvo é dada por uma porcentagem entre:
    \begin{tasks}
        \task 5\% e 15\%.
        \task 15\% e 25\%.
        \task 25\% e 35\%.
        \task 35\% e 45\%.
        \task 45\% e 55\%.
    \end{tasks}
\end{question}

\begin{question}%[concurso:EAM; ano:2020; assunto:; alternativa:]
Para compor a tripulação de um voo, certa companhia de aviação dispõe de 5 pilotos, 3 copilotos, 4 comissários e 6 aeromoças. De quantos modos ela pode escalar uma equipe para um voo, sabendo que esse voo precisa de um piloto, um copiloto, dois comissários e 3 aeromoças?
    \begin{tasks}
        \task 2 140.
        \task 1 920.
        \task 1 800.
        \task 1 750.
        \task 1 280.
    \end{tasks}
\end{question}

\begin{question}%[concurso:EAM; ano:2020; assunto:; alternativa:]
Considere as matrizes \(A\) e \(B\) a seguir:
\[ A = \begin{bmatrix}
        x & 1 \\
        -2 & x 
    \end{bmatrix}
\textrm{ e } 
   B = \begin{bmatrix}
        1 & x \\
        1 & -4 
    \end{bmatrix}
 \]

Existem dois valores \(x_1\) e \( x_2 ~ (x_1 > x_2)\) tal que \(det(A) + det(B) = 0\). É correto afirmar que a expressão \(5x_1 - 3x_2\) é igual a:
    \begin{tasks}
        \task 18.
        \task 13.
        \task 10.
        \task 7.
        \task 6.
    \end{tasks}
\end{question}

\begin{question}%[concurso:EAM; ano:2020; assunto:; alternativa:]
Observe o triângulo a seguir.

INSERIR IMAGEM

No triângulo \(ABC\) traçamos o segmento \(AD\) de forma que \(DC=AC\). Se o ângulo \(B\hat{A}C\) supera em 40\(^\circ\) o ângulo \(ABC\), é correto afirmar que o ângulo \(B\hat{A}D\) mede, em graus:
    \begin{tasks}
        \task 35\(^\circ\).
        \task 30\(^\circ\).
        \task 25\(^\circ\).
        \task 20\(^\circ\).
        \task 15\(^\circ\).
    \end{tasks}
\end{question}

\begin{question}%[concurso:EAM; ano:2020; assunto:; alternativa:]
Para determinar se uma solução é básica, neutra ou ácida calcula-se o potencial hidrogeniônico (Ph) da solução através da fórmula \(PH = -\log [H^{+}]\) onde \(H^{+}\) é a concentração hidrogeniônica da solução. Considere o suco de magnésio com \(H^{+} = 10^{-10}\) e a bile segregada pelo fígado humano com \(H^{+} = 10^{-8}\) e solução classificada por meio dos seguintes parâmetros:

TABELA

Com base nessas informações, é correto afirmar que:
    \begin{tasks}
        \task a bile é básica e o suco de magnésio é ácido.
        \task a bile é ácida e o suco de magnésio é básico.
        \task a bile é básica e o suco de magnésio é básico.
        \task a bile é ácida e o suco de magnésio é ácido.
        \task ambas as soluções são neutras.
    \end{tasks}
\end{question}

\begin{question}%[concurso:EAM; ano:2020; assunto:; alternativa:]
Em um quadrilátero, os ângulos internos são expressos em graus por \(3x + 80, 40 - 3x, 90-5x\) e \(2x + 120\). É correto afirmar que o menor ângulo mede:
    \begin{tasks}
        \task 40\(^\circ\).
        \task 50\(^\circ\).
        \task 60\(^\circ\).
        \task 70\(^\circ\).
        \task 80\(^\circ\).
    \end{tasks}
\end{question}

\begin{question}%[concurso:EAM; ano:2020; assunto:; alternativa:]
Num paralelogramo dois de seus lados adjacentes formam o ângulo de 30\(^\circ\) e medem 5cm e \(5\sqrt{3}\)cm respectivamente. Calcule a diferença entre a diagonal maior e a diagonal menor desse paralelogramo e assinale a opção que apresenta essa diferença.
    \begin{tasks}
        \task \(5(\sqrt{7} - 1)\).
        \task \(5(\sqrt{7} - 2)\).
        \task \(5(\sqrt{3} - 1)\).
        \task \(5\sqrt{3}\).
        \task \(5\sqrt{7}\).
    \end{tasks}
\end{question}

\begin{question}%[concurso:EAM; ano:2020; assunto:; alternativa:]
As raízes do polinômio \(p(x) = x^3 -10x^2 + 29x - 20\) são as dimensões de um paralelepípedo retângulo. É correto afirmar que a área de todas as faces da figura em unidades de área é igual a:
    \begin{tasks}
        \task 28.
        \task 29.
        \task 36.
        \task 48.
        \task 58.
    \end{tasks}
\end{question}

\begin{question}%[concurso:EAM; ano:2020; assunto:; alternativa:]
Um estimativa de dados indica que, caso o preço do ingresso para um jogo de futebol, custe R\$ 20,00, haverá um público de 3 600 pagantes, arrecadando um total de R\$ 72 000,00. Entretanto foi estimado também que, a cada aumento de R\$ 5,00 no preço do ingresso, o público diminuiria em 100 pagantes. Considerando tais estimativas, para que a arrecadação seja a maior possível, o preço unitário do ingresso de tal jogo deve ser:
    \begin{tasks}
        \task R\$ 30,00.
        \task R\$ 60,00.
        \task R\$ 80,00.
        \task R\$ 100,00.
        \task R\$ 120,00.
    \end{tasks}
\end{question}

\begin{question}%[concurso:EAM; ano:2020; assunto:; alternativa:]
Ao resolver a equação \(6445^2 + 3x = 6446^2\), encontramos para \(x\) um número inteiro tal que a soma dos seus algarismos é igual a:
    \begin{tasks}
        \task 14
        \task 18
        \task 22
        \task 26
        \task 28
    \end{tasks}
\end{question}

\begin{question}%[concurso:EAM; ano:2020; assunto:; alternativa:]
No almoxarifado de uma escola, encontram-se numa caixa 60 lápis e 40 canetas, sendo que 24 lápis e 16 canetas são intocados. Ao escolhermos uma peça ao acaso, é correto afirmar que a probabilidade de ser um lápis ou ser um objeto intocado é igual a:
    \begin{tasks}
        \task 84\%.
        \task 76\%.
        \task 60\%.
        \task 50\%.
        \task 36\%.
    \end{tasks}
\end{question}

\begin{question}%[concurso:EAM; ano:2020; assunto:; alternativa:]
Para construir uma ponte entre duas margens de um rio foram marcados, primeiramente, dois pontos \(A\) e \(B\) numa mesma margem distantes 100m e um ponto \(C\) na margem oposta. Utilizando um teodolito (aparelho utilizado para medição de ângulo) descobriram-se as seguintes informações: ângulo \(C\hat{A}B = 30^\circ\) e ângulo \(A\hat{B}C = 75^\circ\). Sabe-se que a ponte deverá ter o menor tamanho possível saindo do ponto \(C\) e chegando a margem oposta. Sendo assim, é correto afirmar que o comprimento dessa ponte será igual a:
    \begin{tasks}
        \task 20m.
        \task 30m.
        \task 40m.
        \task 50m.
        \task 60m.
    \end{tasks}
\end{question}

\begin{question}%[concurso:EAM; ano:2020; assunto:; alternativa:]
Na figura abaixo tem-se um pentágono regular \(ABCDE\) no qual devem ser traçadas as diagonais \(CE\) e \(BD\) e um segmento \(AM\), onde \(M\) é o ponto médio do lado \(CD\). Sabe-se também que \(AM\) passa pelo ponto de intersecção das diagonais traçadas.

INSERIR IMAGEM

Com base nessas informações, é correto afirmar que o número \(n\) de triângulos na figura formada, após os traços, é tal que \(n\) vale:
    \begin{tasks}
        \task 6.
        \task 7.
        \task 8.
        \task 9.
        \task 10.
    \end{tasks}
\end{question}

\begin{question}%[concurso:EAM; ano:2020; assunto:; alternativa:]
Considere a coroa circular formada pelas circunferências \(L_1\) e \(L_2\) cuja soma dos raios vale 0,4dm, conforme figura a seguir.

INSERIR IMAGEM

Se a área da coroa é igual a \(\pi\)dm\(^\circ\), é correto afirmar que a diferença positiva em dm entre os comprimento das circunferências \(L_1\) e \(L_2\) é igual a:
    \begin{tasks}
        \task \(2\pi\).
        \task \(3\pi\).
        \task \(4\pi\).
        \task \(5\pi\).
        \task \(6\pi\).
    \end{tasks}
\end{question}
