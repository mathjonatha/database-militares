\begin{question}%[codigo:EAM202101AMX; concurso:EAM; ano:2021; assunto:; alternativa:]
Dadas as progressões aritméticas \(A: (2,x,8)\), \(B:(5,y,11)\) e \(C:(8,z,14)\). Determine a soma dos seis primeiros termos da \(PA(x,y,z, \ldots)\) e marque a opção correta.
    \begin{tasks}
        \task 15.
        \task 24.
        \task 33.
        \task 65.
        \task 75.
    \end{tasks}
\end{question}

\begin{question}%[codigo:EAM202102AMX; concurso:EAM; ano:2021; assunto:; alternativa:]
Dada a equação \(\frac{p^q - p^{-q}}{p^q + p^{-q}} = r\), onde \(q \in \mathbb{R}\) e \(0 < p \neq 1\), o valor de \(p^{2q}\) é:
    \begin{tasks}
        \task \(\frac{1-r}{r+1}\).
        \task \(r\).
        \task \(\frac{r+1}{1-r}\).
        \task \(r+1\).
        \task \(r-1\).
    \end{tasks}
\end{question}

\begin{question}%[codigo:EAM202103AMX; concurso:EAM; ano:2021; assunto:; alternativa:]
Determine o cosseno de \(1935^\circ\) e marque a opção correta.
    \begin{tasks}
        \task \(\frac{\sqrt{2}}{2}\).
        \task \(1\).
        \task \(\frac{1}{2}\).
        \task \(-\frac{1}{2}\).
        \task \(\frac{- \sqrt{2}}{2}\).
    \end{tasks}
\end{question}

\begin{question}%[codigo:EAM202104AMX; concurso:EAM; ano:2021; assunto:; alternativa:]
Encontre a medida do segmento \(\overline{CD}\) na figura abaixo, sabendo que \(BCDE\) é um retângulo e \(\overline{BA} = 75\)cm, e marque a opção correta.

INSERIR IMAGEM
    \begin{tasks}
        \task 25 cm.
        \task 25\(\sqrt{3}\) cm.
        \task 50 cm.
        \task 75 cm.
        \task 75 \(\sqrt{3}\) cm.
    \end{tasks}
\end{question}

\begin{question}%[codigo:EAM202105AMX; concurso:EAM; ano:2021; assunto:; alternativa:]
Encontre o valor de \(K\) para que o resto da divisão de \(P(x) = 5x^2 - 4kx + 2\) por \(2x-6\) seja \(5\), e marque a opção correta.
    \begin{tasks}
        \task \(\frac{9}{2}\).
        \task \(\frac{7}{2}\).
        \task \(\frac{11}{2}\).
        \task \(\frac{10}{2}\).
        \task \(\frac{12}{2}\).
    \end{tasks}
\end{question}

\begin{question}%[codigo:EAM202106AMX; concurso:EAM; ano:2021; assunto:; alternativa:]
Determine o valor de \(\log_{3\sqrt{3}} 27\) e marque a opção correta.
    \begin{tasks}
        \task 5.
        \task 4.
        \task 3.
        \task 2.
        \task 1.
    \end{tasks}
\end{question}

\begin{question}%[codigo:EAM202107AMX; concurso:EAM; ano:2021; assunto:; alternativa:]
A soma dos ângulos internos do polígono que possui o número de lados igual ao número de diagonais é:
    \begin{tasks}
        \task  90\(^\circ\).
        \task 180\(^\circ\).
        \task 540\(^\circ\).
        \task 560\(^\circ\).
        \task 720\(^\circ\).
    \end{tasks}
\end{question}

\begin{question}%[codigo:EAM202108AMX; concurso:EAM; ano:2021; assunto:; alternativa:]
Assinale a opção que contém o número de anagramas da palavra APRENDIZ.
    \begin{tasks}
        \task 40 300.
        \task 40 320.
        \task 40 330.
        \task 40 340.
        \task 40 350.
    \end{tasks}
\end{question}

\begin{question}%[codigo:EAM202109AMX; concurso:EAM; ano:2021; assunto:; alternativa:]
Em uma loja de eletroeletrônicos, um aparelho de R\$ 1 450,00 na virada do mês, passou a custar R\$ 1 740,00. O preço desse aparelho teve um aumento de:
    \begin{tasks}
        \task 20\%.
        \task 25\%.
        \task 30\%.
        \task 35\%.
        \task 40\%.
    \end{tasks}
\end{question}

\begin{question}%[codigo:EAM202110AMX; concurso:EAM; ano:2021; assunto:função exponencial; alternativa:]
Em uma cidade, a população têm sido contaminada pelo novo Sars-coV-2. Suponha que o número de contaminados pelo vírus seja dado pela função \(f(x) = \left( 10 - \frac{1}{2^x}\right) \cdot 10 000\), onde \(x\) representa a quantidade de meses. Assinale a opção que apresenta o número de contaminados, nessa cidade, no terceiro mês.
    \begin{tasks}
        \task 98 000.
        \task 98 700.
        \task 98 720.
        \task 98 750.
        \task 98 950.
    \end{tasks}
\end{question}

\begin{question}%[codigo:EAM202111AMX; concurso:EAM; ano:2021; assunto:; alternativa:]
Se as matrizes \(A = (a_{ij}), B = (b_{ij})\) e \(C = (c_{ij})\), ambas quadradas e de 3\textordfeminine{} ordem, estão definidas:

\(A=\begin{cases}
    i^j, \textrm{ se } i> j\\
    i + j, \textrm{ se } i = j\\
    -i, \textrm{ se } i < j
\end{cases} , B = b_{ij = i^2}\) e \(C = A + B\). Nesse caso, o cofator de \(C_{32}\) é:
    \begin{tasks}
        \task -18.
        \task -6.
        \task -1.
        \task 6.
        \task 18.
    \end{tasks}
\end{question}

\begin{question}%[codigo:EAM202112AMX; concurso:EAM; ano:2021; assunto:; alternativa:]
Dada uma função exponencial \(f(x) = a^x\), a respeito de suas características é correto afirmar que a função é:
    \begin{tasks}
        \task decrescente para a base \(a\) maior que \(1 (a>1)\).
        \task crescente para \(x\) maior que \(0\).
        \task crescente se a base \(a\) for igual a \(1(a=1)\).
        \task crescente para \(x\) maior que \(0\) e menor \(1(0<x<1)\).
        \task decrescente para a base \(a\) maior que \(0\) e menor que \(1(0<a<1)\).
    \end{tasks}
\end{question}

\begin{question}%[codigo:EAM202113AMX; concurso:EAM; ano:2021; assunto:; alternativa:]
Para qualquer \(a\) real, a expressão: \(4^a + 4^{a+1} + (4^a \cdot 16) + 4^{a+3} + 4^a \cdot 256 + 4^{a \div 5}\) é equivalente a:
    \begin{tasks}
        \task \( 4^{6a} + 15\).
        \task \(4^a + 15\).
        \task \(1365^a\).
        \task \(1365 \cdot 4^a\).
        \task \(1365^{2a}\).
    \end{tasks}
\end{question}

\begin{question}%[codigo:EAM202114AMX; concurso:EAM; ano:2021; assunto:; alternativa:]
Uma pesquisa de mercado sobre o consumo de três marcas de café \(A,B\) e \(C\), apresentou os seguintes resultados:

\begin{itemize}
    \item 60\% consomem o produto \(A\);
    \item 51\% consomem o produto \(B\);
    \item 15\% consomem o produto \(C\);
    \item 5\% consomem os três produtos;
    \item 11\% consomem os produtos \(A\) e \(B\); e
    \item 10\% consomem os produtos \(B\) e \(C\).
\end{itemize}
Qual é o percentual relativo à quantidade de pessoas que consomem, simultaneamente, os produtos \(A\) e \(C\) sem consumir o \(B\)?
    \begin{tasks}
        \task 3\%.
        \task 5\%.
        \task 7\%.
        \task 9\%.
        \task 11\%.
    \end{tasks}
\end{question}

\begin{question}%[codigo:EAM202115AMX; concurso:EAM; ano:2021; assunto:; alternativa:]
Determine a área hachurada, no gráfico abaixo, sabendo que \(V\) é o vértice da parábola, e marque a opção correta.

INSERIR IMAGEM
    \begin{tasks}
        \task 40.
        \task 50.
        \task 60.
        \task 70.
        \task 80.
    \end{tasks}
\end{question}
