\begin{question}%[concurso:EAM; ano:2022; assunto:; alternativa:]
Observe a figura abaixo:
INSERIR FIGURA
Se \(ABCD\) é um quadrado e \(ABP\) um triângulo equilátero, determine o ângulo \(x\) e assinale a opção correta.
    \begin{tasks}
        \task \(135^\circ\).
        \task \(105^\circ\).
        \task \(100^\circ\).
        \task \(97^\circ\).
        \task \(95^\circ\).
    \end{tasks}
\end{question}

\begin{question}%[concurso:EAM; ano:2022; assunto:probabilidade; alternativa:]
Uma das sensações nos jogos online é o \textit{Call of Duty - WARZONE}, pois, em um dos seus modos de jogo a equipe vencedora é a última que sobrevive. Considera um jogador do \textit{WARZONE} chamado NEGUEBA. Suponto que em uma partido online no \textit{WARZONE} existam sempre 4 caminhos para tentar derrubar um oponente, sendo que em apenas um deles é possível derrubar. Assim, para cada caminho, NEGUEBA tem probabilidade de \(\frac{1}{4}\) de escolher o o que vai derrubar um oponente se ele está adivinhando e 1 se ele sabe esse caminho. NEGUEBA sabe 10\% dos caminhos para derrubar um oponente. Se ele derrubou um dos oponentes, qual é a probabilidade dele ter adivinhado o caminho?
    \begin{tasks}
        \task \(\frac{9}{13}\).
        \task \(\frac{4}{5}\).
        \task \(\frac{8}{13}\).
        \task \(\frac{7}{16}\).
        \task \(\frac{3}{7}\).
    \end{tasks}
\end{question}

\begin{question}%[concurso:EAM; ano:2022; assunto:; alternativa:]
Determine a equação reduzida da elipse cujo eixo menor tem por extremos os focos da hipérbole \(x^2 - y^2 = -1\) e cuja excentricidade é o inverso da excentricidade da hipérbole dada, como mostra a figura abaixo, e assinale a opção correta.
INSERIR IMAGEM
    \begin{tasks}
        \task \(\frac{x^2}{4} + \frac{y^2}{2} = 1 \).
        \task \(\frac{x^2}{3} + \frac{y^2}{2} = 1 \).
        \task \(\frac{x^2}{2} + \frac{y^2}{4} = 1 \).
        \task \(\frac{x^2}{2} + \frac{y^2}{3} = 1 \).
        \task \(x^2 + y^2 = 1\).
    \end{tasks}
\end{question}

\begin{question}%[concurso:EAM; ano:2022; assunto:; alternativa:]
Assinale a opção que apresenta a soma de todos os inteiros que divididos por 11 dão resto 7 e estão compreendidos entre 200 e 400.
    \begin{tasks}
        \task 5373.
        \task 5431.
        \task 5578.
        \task 5691.
        \task 5743.
    \end{tasks}
\end{question}

\begin{question}%[concurso:EAM; ano:2022; assunto:; alternativa:]
As arestas laterais de uma pirâmide medem 52cm e sua base é um triângulo isósceles cujos lados medem 24cm, \(12\sqrt{10}\)cm e \(12\sqrt{10}\)cm. Sabendo que a projeção do vértice da pirâmide na base triangular é o centro de sua circunferência circunscrita, determine a altura dessa pirâmide e assinale a opção correta.
    \begin{tasks}
        \task 12cm.
        \task 16cm.
        \task 30cm.
        \task 36cm.
        \task 48cm.
    \end{tasks}
\end{question}

\begin{question}%[concurso:EAM; ano:2022; assunto:; alternativa:]
Considere a elipse \(E\) com centro na origem, um dos focos em \(F_1 \left( 0, \sqrt{\frac{2}{3}} \right)\) e que passa pelo ponto \(P \left( \frac{1}{2}, \frac{1}{2} \right)\), como mostrado na figura abaixo. Assinale a opção correta que apresenta a excentricidade de \(E\).
INSERIR IMAGEM
    \begin{tasks}
        \task \(\frac{1}{6}\).
        \task \(\frac{1}{2}\).
        \task \(\sqrt{\frac{2}{3}}\).
        \task \(1\).
        \task \(\sqrt{\frac{3}{2}}\).
    \end{tasks}
\end{question}

\begin{question}%[concurso:EAM; ano:2022; assunto:; alternativa:]
Um nutricionista deseja preparar uma refeição diária equilibrada em vitaminas \(A,B\) e \(C\). Para isso ele dispõe de 3 tipos de alimentos \(X,Y\) e \(Z\). O alimento \(X\) possui uma unidade de vitamina \(A\), 10 unidades de vitamina \(B\) e uma unidade de vitamina \(C\). O alimento \(Y\) possui 9 unidades de vitamina \(A\), uma de vitamina \(B\) e uma unidade de vitamina \(C\). O alimento \(Z\) possui 2 unidades de vitamina \(A\), 2 unidades de vitamina \(B\) e 2 unidades de vitamina \(C\). Sabendo que para uma alimentação diária equilibrada em vitamina deve conter 160 unidade de vitamina \(A\), 170 unidades de vitamina \(B\) e 140 unidades de vitamina \(C\), calcule a soma das quantidades de alimentos que deverão ser utilizadas na refeição e assinale a opção correta.
    \begin{tasks}
        \task 45.
        \task 50.
        \task 55.
        \task 60.
        \task 65.
    \end{tasks}
\end{question}

\begin{question}%[concurso:EAM; ano:2022; assunto:; alternativa:]
Encontre os valores dos arcos \(x\) e \(y\) indicados na figura abaixo e assinale a opção correta.
INSERIR IMAGEM
    \begin{tasks}
        \task \(x = 30^\circ\) e \(y = 90^\circ\).
        \task \(x = 45^\circ\) e \(y = 90^\circ\).
        \task \(x = 45^\circ\) e \(y = 75^\circ\).
        \task \(x = 60^\circ\) e \(y = 75^\circ\).
        \task \(x = 90^\circ\) e \(y = 60^\circ\).
    \end{tasks}
\end{question}

\begin{question}%[concurso:EAM; ano:2022; assunto:; alternativa:]
Uma esfera com centro em \(O\) possui volume igual a \(\frac{1372\pi}{3}\) cm\(^2\). Se tomarmos um plano e o fizermos interceptar essa esfera a uma distância \(d\) do seu centro, a seção plana circular resultante, de centro \(O'\), terá área igual a \(24\pi\) cm\(^2\) (figura abaixo). Assim, de acordo com os dados, calcule o valor de \(d\), ou seja \(\overline{OO'}\), e assinale a opção correta.

INSERIR IMAGEM
    \begin{tasks}
        \task 1cm.
        \task 3cm.
        \task 5cm.
        \task 7cm.
        \task 10cm.
    \end{tasks}
\end{question}

\begin{question}%[concurso:EAM; ano:2022; assunto:; alternativa:]
Considere duas fontes de luz, \(A\) e \(B\), situadas no eixo das abcissas, com \(A\) na origem. A fonte \(B\) é 4 vezes mais brilhante do que a fonte \(A\) e distam 15m entre si. Suponha que um objeto \(C\) é posto no eixo das abcissas entre \(A\) e \(B\). Sabendo que a luminosidade em \(C\) é diretamente proporcional à intensidade da fonte e inversamente proporcional ao quadrado da distância desse ponto à mesma fonte. A que distância de \(A\) deve estar \(C\) para que seja iluminado igualmente por ambas as fontes?
    \begin{tasks}
        \task 1m.
        \task 3m.
        \task 5m.
        \task 6m.
        \task 7m.
    \end{tasks}
\end{question}

\begin{question}%[concurso:EAM; ano:2022; assunto:; alternativa:]
Assinale a opção que apresenta o valor de \(x\) para o qual é solução da equação \(\log_9 x + \log_{27} x - \log_3 x = -1\).
    \begin{tasks}
        \task 603.
        \task 729.
        \task 831.
        \task 867.
        \task 906.
    \end{tasks}
\end{question}

\begin{question}%[concurso:EAM; ano:2022; assunto:; alternativa:]
Calcule a área \(S\) e o perímetro \(P\) do triângulo \(ABA'\) abaixo e assinale a opção correta.

INSERIR IMAGEM
    \begin{tasks}
        \task \(S = \sqrt{2}\) e \(P=1+ \sqrt{3}\).
        \task \(S = \sqrt{3}\) e \(P=5 + \sqrt{2}\).
        \task \(S = 5\sqrt{2}\) e \(P=\sqrt{3}\).
        \task \(S = 8\sqrt{3}\) e \(P=4(3+ \sqrt{3})\).
        \task \(S = 10\sqrt{3}\) e \(P=2(2+\sqrt{3})\).
    \end{tasks}
\end{question}

\begin{question}%[concurso:EAM; ano:2022; assunto:; alternativa:]
Sabendo que a reta \(r\) é determinada pelos pontos de interseção da função \(f(x)= x^2-x\) com a sua inversa \(f^{-1}(x)\), como representado na figura abaixo, e seja o menor segmento de reta \(PP'\) que une o ponto \(P(10,0)\) a esta reta, com \(P'\in r\). Considere o triângulo retângulo \(OP'P\) sendo \(O\) a origem do eixo cartesiano e reto em \(P'\). Desse modo, encontre o tamanho do segmento \(PP'\) e assinale a opção correta.

INSERIR IMAGEM
    \begin{tasks}
        \task \(\sqrt{2}\).
        \task \(\sqrt{3}\).
        \task \(2\sqrt{3}\).
        \task \(5\sqrt{2}\).
        \task \(5\sqrt{3}\).
    \end{tasks}
\end{question}

\begin{question}%[concurso:EAM; ano:2022; assunto:; alternativa:]
Um vídeo game é vendido à vista por R\$ 2 000,00 ou a prazo com R\$ 400,00 de entrada e mais uma parcela de R\$ 1 800,00 quatro meses após a compra. Assinale a opção que apresenta a taxa mensal de juros compostos do financiamento. Considere apenas 3 casas decimais e sem arredondamento.
    \begin{tasks}
        \task 2,3\%.
        \task 2,9\%.
        \task 3,3\%.
        \task 4,0\%.
        \task 4,4\%.
    \end{tasks}
\end{question}

\begin{question}%[concurso:EAM; ano:2022; assunto:; alternativa:]
Sabe-se que \((1 - \cos^2(x))(\cotg^2(x) + 1) = A\) para \(x\) diferente de \(k\pi\), com \(k \in \mathbb{Z}\), e que \(\frac{\sec^(x) - 1}{\tg^2(x) + 1} = B\), quando \(\sen(x) = \frac{\sqrt{2}}{2}\). Assim, assinale a opção que apresenta o valor de \(B^A\).
    \begin{tasks}
        \task \(0\).
        \task \(\frac{1}{2}\).
        \task \(1\).
        \task \(\frac{3}{2}\).
        \task \(2\).
    \end{tasks}
\end{question}
