\begin{question}%[codigo:EAM201701AMX; concurso:EAM; ano:2017; assunto:; alternativa:]
Sendo \(x - \frac{2}{x} = a\), então \(x^2 + \frac{4}{x}\) é igual a:
    \begin{tasks}
        \task \(a^2 + 4\).
        \task \(a^2 - 4\).
        \task \(a^2\).
        \task \(a + 4\).
        \task \(a - 4\).
    \end{tasks}
\end{question}

\begin{question}%[codigo:EAM201702AMX; concurso:EAM; ano:2017; assunto:; alternativa:]
Analise a figura a seguir.

INSERIR FIGURA

Calcule a soma das áreas hachuradas da figura acima, sabendo que os polígonos I e II são quadrados, e assinale a opção correta.
    \begin{tasks}
        \task \(22\sqrt{3}\).
        \task \(22\).
        \task \(13 + 4\sqrt{3}\).
        \task \(11\).
        \task \(11\sqrt{3}\).
    \end{tasks}
\end{question}

\begin{question}%[codigo:EAM201703AMX; concurso:EAM; ano:2017; assunto:; alternativa:]
Observe a figura a seguir.

INSERIR FIGURA

Sabendo que, na figura acima, as reatas \(r\) e \(s\) são paralelas, é correto afirmar que o valor de \(x\) é igual a:
    \begin{tasks}
        \task 90\(^\circ\).
        \task 85\(^\circ\).
        \task 80\(^\circ\).
        \task 75\(^\circ\).
        \task 70\(^\circ\).
    \end{tasks}
\end{question}

\begin{question}%[codigo:EAM201704AMX; concurso:EAM; ano:2017; assunto:; alternativa:]
Deseja-se azulejar, até o teto, as 4 paredes de uma cozinha. Sabe-se que a cozinha possui 2 portas medindo 210cm de altura e 80cm de largura cada uma, e uma janela com 150cm de altura e 110cm de comprimento. O comprimento, a largura e a altura da cozinha são iguais a 5,0m, 4,0m e 3,0m, respectivamente. Determine o número mínimo de metros quadrados inteiros de azulejos que devem ser comprados a assinale a opção correta.
    \begin{tasks}
        \task 42.
        \task 43.
        \task 49.
        \task 55.
        \task 58.
    \end{tasks}
\end{question}

\begin{question}%[codigo:EAM201705AMX; concurso:EAM; ano:2017; assunto:; alternativa:]
Considerando \(n(P)\) como a notação que determina o número de elementos de um conjunto \(P,A \times X\) como o produto cartesiano entre dois conjuntos finitos A e B e sabendo-se ainda que \(n(A) = 2x - 3, n(B) = x-5\) e \(n(A \times B)= x^2 +10x -27\), é correto afirmar que o valor numérico de \(x\) é:
    \begin{tasks}
        \task um número primo.
        \task um múltiplo de 5.
        \task um múltiplo de 7.
        \task um múltiplo de 11.
        \task um múltiplo de 13.
    \end{tasks}
\end{question}

\begin{question}%[codigo:EAM201706AMX; concurso:EAM; ano:2017; assunto:; alternativa:]
Seja a função real \(f\) definida por \(f(x) = \frac{x+k}{p}\). Sabendo-se que \(f(3) = 2\) e \(f(5) = 4\), determine o valor de \(k+p\) e assinale a opção correta.
    \begin{tasks}
        \task 0.
        \task 1.
        \task 2.
        \task 3.
        \task 4.
    \end{tasks}
\end{question}

\begin{question}%[codigo:EAM201707AMX; concurso:EAM; ano:2017; assunto:; alternativa:]
Sabendo-se que \(A\) e \(B\) são subconjuntos finitos de \(U\), que \( \bar{A}\) é a notação para a operação complementar de \(A\) em relação a \(U\), que \(\bar{A} = \{q,r,s,t,u\}, A \cap B = \{o,p\}\) e \(A \cup B = \{m,n,o,p,q,r\}\), é correto afirmar que:
    \begin{tasks}
        \task \(A\) tem dois elementos e \(B\) tem quatro elementos.
        \task \(A\) tem quatro elementos e \(B\) tem dois elementos.
        \task \(A\) tem três elementos e \(B\) tem três elementos.
        \task \(A\) tem quatro elementos e \(B\) tem quatro elementos.
        \task \(A\) tem um elemento e \(B\) tem cinco elementos.
    \end{tasks}
\end{question}

\begin{question}%[codigo:EAM201708AMX; concurso:EAM; ano:2017; assunto:; alternativa:]
Sabendo que a fração \(\frac{y}{4}\) é proporcional à fração \(\frac{3}{6-2\sqrt{3}}\), é correto afirmar que \(y\) é igual a:
    \begin{tasks}
        \task \( 1 - 2\sqrt{3}\).
        \task \( 6 + 3\sqrt{3}\).
        \task \( 2 - \sqrt{3}\).
        \task \( 4 + 3\sqrt{3}\).
        \task \( 3+ \sqrt{3}\).
    \end{tasks}
\end{question}

\begin{question}%[codigo:EAM201709AMX; concurso:EAM; ano:2017; assunto:; alternativa:]
A soma de um número \(x\) com o dobro de um número \(y\) é \(-7\); e a diferença entre o triplo desse número \(x\) e o número \(y\) é igual a \(7\). Sendo assim, é correto afirmar que o produto \(xy\) é igual a:
    \begin{tasks}
        \task -15.
        \task -12.
        \task -10.
        \task -4.
        \task -2.
    \end{tasks}
\end{question}

\begin{question}%[codigo:EAM201710AMX; concurso:EAM; ano:2017; assunto:; alternativa:]
O número natural \(N= 2^3 \cdot 3^p\) possui 20 divisores positivos. Sendo assim, o valor de \(p\) é:
    \begin{tasks}
        \task 2.
        \task 3.
        \task 4.
        \task 5.
        \task 6.
    \end{tasks}
\end{question}

\begin{question}%[codigo:EAM201711AMX; concurso:EAM; ano:2017; assunto:; alternativa:]
Apoiado em dois pilares construídos sobre um terreno plano e distantes 3m um do outro, constrói-se um telhado, cuja inclinação é de 30\(^\circ\) em relação ao piso. Se o pular de menor altura mede 4 metros, qual é a altura do outro pilar? Dado: \(\sqrt{3} = 1,7\)
    \begin{tasks}
        \task 5,5m.
        \task 5,7m.
        \task 6,0m.
        \task 6,5m.
        \task 6,9m.
    \end{tasks}
\end{question}

\begin{question}%[codigo:EAM201712AMX; concurso:EAM; ano:2017; assunto:; alternativa:]
Um colecionador de selos criou um catálogo de selos em uma pasta com 20 páginas numeradas de 1 até 20, cada uma com 15 selos, distribuídos em 5 linhas e 3 colunas. Os selos foram numerados de 1 a 300. Nesse catálogo, alguns selos são considerados raros e ocupam as posições 9\textordfeminine{},18\textordfeminine{},27\textordfeminine{},36\textordfeminine{} e assim sucessivamente. Depois que o catálogo for completado com todos os selos, é correto afirmar que o número de última página que terminará com um selo raro será
    \begin{tasks}
        \task 9.
        \task 11.
        \task 12.
        \task 18.
        \task 20.
    \end{tasks}
\end{question}

\begin{question}%[codigo:EAM201713AMX; concurso:EAM; ano:2017; assunto:; alternativa:]
No dia 17-10-2016, á zero hora, iniciou-se mais uma vez o horário de verão no Rio de Janeiro, que tem sido usado com objetivo de economizar energia elétrica nos momentos de pico e evitar sobrecarga no sistema. No dia 16-10-2016, um avião partiu de St. John's, Canadá, com destino ao Rio de Janeiro. A saída aconteceu às 21h e 45min e o voo teve duração de 13h e 45min. Considerando que entre St. John's e Rio de Janeiro não há diferença de fuso horário, a que horas local o avião chegou ao Rio de Janeiro?
    \begin{tasks}
        \task 9h e 30min.
        \task 10h e 30min.
        \task 11h e 15min.
        \task 11h e 45min.
        \task 12h e 30min.
    \end{tasks}
\end{question}

\begin{question}%[codigo:EAM201714AMX; concurso:EAM; ano:2017; assunto:; alternativa:]
Observe a figura a seguir.

INSERIR IMAGEM

Na figura acima, tem-se um triângulo isósceles \(ACD\), no qual o segmento \(\overline{AB}\) mede 3cm, o lado desigual \(AD\) mede \(10\sqrt{2}\)cm e os segmentos \(\overline{AC}\) e \(\overline{CD}\) são perpendiculares. Sendo assim, é correto afirmar que o segmento \(\overline{BD}\) mede:
    \begin{tasks}
        \task \(\sqrt{53}\)cm.
        \task \(\sqrt{97}\)cm.
        \task \(\sqrt{111}\)cm.
        \task \(\sqrt{149}\)cm.
        \task \(\sqrt{161}\)cm.
    \end{tasks}
\end{question}

\begin{question}%[codigo:EAM201715AMX; concurso:EAM; ano:2017; assunto:; alternativa:]
A área de um retângulo corresponde à expressão \(K^2-10k-24\) quando \(k=36\). Sendo assim, calcule suas dimensões e assinale a opção correta.
    \begin{tasks}
        \task 38 e 24.
        \task 36 e 32.
        \task 63 e 24.
        \task 54 e 38.
        \task 32 e 24.
    \end{tasks}
\end{question}

\begin{question}%[codigo:EAM201716AMX; concurso:EAM; ano:2017; assunto:; alternativa:]
Observe a figura abaixo.

INSERIR FIGURA

Um prédio projeta no solo uma sombra de 30m de extensão no mesmo instante em que uma pessoa de 1,80m projeta uma sombra de 2,0m. Pode-se afirmar que a altura do prédio vale
    \begin{tasks}
        \task 27m.
        \task 30m.
        \task 33m.
        \task 36m.
        \task 40m.
    \end{tasks}
\end{question}