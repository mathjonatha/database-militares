\begin{question}%[concurso:EAM; ano:2018; assunto:; alternativa:]
A partir de um dos vértices de um polígono convexo pode-se traçar tantas diagonais quantas são o total de diagonais de um pentágono. É correto afirmar que esse polígono é um:
    \begin{tasks}
        \task Hexágono.
        \task Heptágono.
        \task Octógono.
        \task Decágono.
        \task Dodecágono.
    \end{tasks}
\end{question}

\begin{question}%[concurso:EAM; ano:2018; assunto:; alternativa:]
Considere a função \(f(x) = k \cos(x)\), onde \(K\) é uma constante real, diferente de zero, e \(x\) é valor de graus. É correto afirmar que a razão entre \(f(60^\circ)\) e \(f(45^\circ)\) é igual a:
    \begin{tasks}
        \task \(\frac{\sqrt{2}}{2}\).
        \task \(\frac{1}{2}\).
        \task \(\frac{\sqrt{3}}{2}\).
        \task \(\frac{\sqrt{2}}{3}\).
        \task \(2\).
    \end{tasks}
\end{question}

\begin{question}%[concurso:EAM; ano:2018; assunto:; alternativa:]
Observe a figura abaixo.

INSERIR FIGURA

Uma piscina se utiliza das duas torneiras e do ralo da figura acima para manutenção do seu nível de água. A torneira \(B\), aberta sozinha, enche a piscina em 6 horas e torneira \(A\), também sozinha, enche a piscina é 4 horas. Caso a piscina esteja cheia, o ralo esvaziará num tempo \(t\). Num certo dia, o piscineiro, estando a piscina vazia, abriu as duas torneiras, porém esqueceu de fechar o ralo constatando posteriormente que a piscina ficou completamente cheia, nessas condições, em 12 horas. Sendo assim, é correto afirmar que essa piscina com as duas torneiras fechadas e o ralo aberto, estando totalmente cheia, necessitará de \(t\) horas para esvaziá-la, sendo \(t\) igual a:
    \begin{tasks}
        \task 3.
        \task 5.
        \task 7.
        \task 9.
        \task 12.
    \end{tasks}
\end{question}

\begin{question}%[concurso:EAM; ano:2018; assunto:; alternativa:]
É correto afirmar que o valor da soma das raízes reais da equação \(x^4 = 7x^2 + 18\) é um número:
    \begin{tasks}
        \task primo.
        \task divisor de 36.
        \task múltiplo de 3.
        \task divisor de 16.
        \task divisor de 25.
    \end{tasks}
\end{question}

\begin{question}%[concurso:EAM; ano:2018; assunto:; alternativa:]
Se a soma dos quadrados das raízes da equação \(x^2 + px + 10 = 0\) é igual a \(29\), é correto afirmar que o valor de \(p^2\) é um múltiplo de:
    \begin{tasks}
        \task 2.
        \task 3.
        \task 5.
        \task 7.
        \task 9.
    \end{tasks}
\end{question}

\begin{question}%[concurso:EAM; ano:2018; assunto:; alternativa:]
Analise a figura a seguir.

INSERIR FIGURA

Na figura, \(AB = AC, BX=BY\) e \(CZ=CY\). Se o ângulo \(A\) mede 40\(^\circ\), então o ângulo \(XYZ\) mede:
    \begin{tasks}
        \task 40\(^\circ\).
        \task 50\(^\circ\).
        \task 60\(^\circ\).
        \task 70\(^\circ\).
        \task 90\(^\circ\).
    \end{tasks}
\end{question}

\begin{question}%[concurso:EAM; ano:2018; assunto:; alternativa:]
Analise a figura abaixo.

INSERIR IMAGEM

A área do trapézio da figura acima é 12. Considere que o segmento \(EC = 4; CD = 2\) e \(GH = 2r\). Considere, ainda, que os pontos \(C,G\) e \(H\) são pontos de tangência e \(r\) é o raio do semicírculo sombreado. Sendo assim, é correto afirmar que a área do semicírculo sombreado é igual a:
    \begin{tasks}
        \task \(\pi\)
        \task \(2\pi\)
        \task \(3\pi\)
        \task \(4\pi\)
        \task \(5\pi\)
    \end{tasks}
\end{question}

\begin{question}%[concurso:EAM; ano:2018; assunto:; alternativa:]
Analise a figura a seguir.

INSERIR FIGURA

Um arquiteto pretende fixar em um painel de 40m de comprimento horizontal sete gravuras com 4m de comprimento horizontal cada. a distância entre duas gravuras consecutivas é \(d\), enquanto que a distância da primeira e da última gravura até as respectivas laterais do painel é \(2d\). Sendo assim, é correto afirmar que \(d\) é igual a:
    \begin{tasks}
        \task 0,85m.
        \task 1,15m.
        \task 1,20m.
        \task 1,25m.
        \task 1,35m.
    \end{tasks}
\end{question}

\begin{question}%[concurso:EAM; ano:2018; assunto:; alternativa:]
Analise as afirmativas abaixo:

\begin{enumerate}[label=\Roman*.]
    \item Todo quadrado é um losango.
    \item Todo quadrado é um retângulo.
    \item Todo retângulo é um paralelogramo.
    \item Todo triângulo equilátero é isósceles.
\end{enumerate}

Assinale a opção correta.
    \begin{tasks}
        \task Apenas a afirmativa I é verdadeira.
        \task As afirmativas I,II,III e IV são verdadeiras.
        \task Apenas as afirmativas I, II e III são verdadeiras.
        \task Apenas as afirmativas III e IV são verdadeiras.
        \task Apenas a afirmativa II é verdadeira.
    \end{tasks}
\end{question}

\begin{question}%[concurso:EAM; ano:2018; assunto:; alternativa:]
A expressão \(\frac{\frac{x}{2x-1}-1}{1+ \frac{x}{1-2x}}\) para \(x \neq 1, x \neq \frac{1}{2}\) e \(x \neq -\frac{1}{2}\) é igual a:
    \begin{tasks}
        \task -2.
        \task -1.
        \task 0.
        \task 2.
        \task 3.
    \end{tasks}
\end{question}

\begin{question}%[concurso:EAM; ano:2018; assunto:; alternativa:]
Se \(A=\sqrt{\sqrt{6}-2} \cdot \sqrt{2+\sqrt{6}}\), então o valor de \(A^2\) é:
    \begin{tasks}
        \task 1.
        \task 2.
        \task 4.
        \task 6.
        \task 36.
    \end{tasks}
\end{question}

\begin{question}%[concurso:EAM; ano:2018; assunto:; alternativa:]
Uma padaria produz 800 pães e, para essa produção, necessita de 12 litros de leite. Se a necessidade de leite é proporcional à produção, se o dono quer aumentar a produção de pães em 25\% e se o litro de leite custa R\$ 2,50, quanto o dono deverá gastar a mais com a compra de leite para atingir sua meta?
    \begin{tasks}
        \task R\$ 5,00.
        \task R\$ 7,50.
        \task R\$ 20,00.
        \task R\$ 30,00.
        \task R\$ 37,50.
    \end{tasks}
\end{question}

\begin{question}%[concurso:EAM; ano:2018; assunto:; alternativa:]
Sabendo-se que \(x - \frac{1}{x} = 1\) é correto afirmar que \(x^3 - \frac{1}{x^3}\) é igual a:
    \begin{tasks}
        \task 1.
        \task 4.
        \task 8.
        \task 12.
        \task 27.
    \end{tasks}
\end{question}

\begin{question}%[concurso:EAM; ano:2018; assunto:; alternativa:]
Dentre os inscritos em um concurso público, 60\% são homens e 40\% são mulheres. Sabe-se que já estão empregados 80\% dos homens e 30\% das mulheres. Qual a porcentagem dos candidatos que já tem emprego?
    \begin{tasks}
        \task 60\%.
        \task 40\%.
        \task 30\%.
        \task 24\%.
        \task 12\%.
    \end{tasks}
\end{question}

\begin{question}%[concurso:EAM; ano:2018; assunto:; alternativa:]
Considerando-se todos os divisores naturais de 360, quantos NÃO são pares?
    \begin{tasks}
        \task 6.
        \task 5.
        \task 4.
        \task 3.
        \task 2.
    \end{tasks}
\end{question}