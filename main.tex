\documentclass{scrartcl} % https://texdoc.org/serve/scrartcl/0
\KOMAoptions{%
    fontsize=11pt,%
    titlepage=off,%
    headings=big,%
    headsepline=off,%
    paper=a4,%
    twoside=off,%
    % parskip=half+,%
    % bibliography=totoc,%
}%
\usepackage[T1]{fontenc}
\usepackage[utf8]{inputenc}
\usepackage[brazil]{babel}
\usepackage{geometry}
\geometry{
	paper=a4paper, 
	top=3cm, 
	bottom=2.5cm, 
	left=1.5cm, 
	right=2cm,
	headheight=14pt, % Header height
	footskip=1.4cm, % Espaço da margem inferior à linha de base do rodapé
	headsep=10pt, % Espaço da margem superior até a linha de base do cabeçalho
	%showframe, % Uncomment to show how the type block is set on the page
}

\title{BANCO DE QUESTÕES MILITARES}
\author{Matheus Jonatha}
% \date{January 2023}

\usepackage{amsmath,amsfonts,amssymb,amsthm,enumitem,multicol}
\everymath{\displaystyle}
\usepackage{tasks}
\usepackage{exsheets}
\SetupExSheets[question]{name=QUESTÃO}

% NOVOS COMANDOS
% Operadores Matemáticos
\newcommand{\sen}{\operatorname{sen}}
\newcommand{\tg}{\operatorname{tg}}
\newcommand{\cotg}{\operatorname{cotg}}
\newcommand{\cossec}{\operatorname{cossec}}

% EAM200401AMD
% EAM -> NOME DO CONCURSO.
% 2004 -> ANO DO CONCURSO/PROVA.
% 01 -> NÚMERO DA QUESTÃO NA PROVA.
% A -> PODE SER: AM,AZ,RO,VE (AMARELA, AZUL, ROSA, VERDE).
% D -> PODE SER: F,M,D (FÁCIL, MÉDIO, DIFÍCIL).

\begin{document}

\maketitle
\begin{multicols}{2}
\begin{question}%[concurso:EAM; ano:2004; assunto:função do segundo grau,função quadrática, funções do segundo grau; alternativa:A]
O lucro mensal de uma fábrica é dado por \(L(x) = -2x^2 + 32x - 56\), sendo \(x\) medido em milhares de peças fabricadas e \(L\) em milhões de Reais. Quando o lucro é nulo, isto é, \(-2x^2 + 32x - 56 = 0\), a quantidade de peças produtivas é a solução positiva da equação, multiplicada por mil, então a quantidade de peças para que o lucro seja nulo é:
    \begin{tasks}
        \task 2 000 ou 14 000.
        \task 3 000 ou 16 000.
        \task 4 000 ou 12 000.
        \task 5 000 ou 16 000.
        \task 7 000 ou 18 000.
    \end{tasks}
\end{question}

\begin{question}%[concurso:EAM; ano:2004; assunto:geometria; alternativa:E]
Na figura, os segmentos \(\overline{AB}, \overline{BC}, \overline{CD}, \overline{DE}\) são respectivamente paralelos aos segmentos \(\overline{MN}, \overline{NQ}, \overline{QO}, \overline{OP}\), o ângulo \(P\hat{O}Q = 35^\circ\) e \(A\hat{B}C = 40^\circ\). O valor do ângulo \(B\hat{C}D\) é:

INSERIR IMAGEM
    \begin{tasks}
        \task \(35^\circ\).
        \task \(40^\circ\).
        \task \(50^\circ\).
        \task \(55^\circ\).
        \task \(75^\circ\).
    \end{tasks}
\end{question}

\begin{question}%[concurso:EAM; ano:2004; assunto:regra de três; alternativa:C]
Se uma torneira enche um reservatório de água de \(5,4\) m\(^3\) a uma razão de 15 litros por minuto, quanto tempo levará para encher completamente o reservatório?
    \begin{tasks}
        \task quatro horas.
        \task cinco horas e meia.
        \task seis horas.
        \task seis horas e meia.
        \task sete horas.
    \end{tasks}
\end{question}

\begin{question}%[concurso:EAM; ano:2004; assunto:gráfico,pizza,interpretação de gráficos; alternativa:A]
Num trabalho de pesquisa feito com \(10 000\) fumantes, divididos em 5 grupos em que a cada grupo foi aplicada uma arma contra fumo, conforme o gráfico abaixo. Sabe-se que 40\% do grupo que utilizaram a acupuntura parou de fumar. O número de pessoas que participaram dessa pesquisa e que pararam de fumar através da acupuntura é:

INSERIR IMAGEM
    \begin{tasks}
        \task \(840\).
        \task \(860\).
        \task \(1020\).
        \task \(1400\).
        \task \(1480\).
    \end{tasks}
\end{question}

\begin{question}%[concurso:EAM; ano:2004; assunto:área, geometria; alternativa:B]
A área da figura hachurada, onde todas as medidas são em metros é: Considere \(\pi = 3,1\) e \(\sqrt{3} = 1,7\)
INSERIR IMAGEM
    \begin{tasks}
        \task \(54,1\).
        \task \(56,1\).
        \task \(58,2\).
        \task \(60,1\).
        \task \(61,3\).
    \end{tasks}
\end{question}

\begin{question}%[concurso:EAM; ano:2004; assunto:; alternativa:B]
No painel o desenho de uma árvore de natal, na forma de um triângulo isósceles, onde a altura, e a base são números inteiros e os lados medem \(\sqrt{10}\), será revestido com um papel de parede, que custa R\$ 8,00 o metro quadrado. Qual o custo mínimo para revestir essa árvore?

INSERIR IMAGEM
    \begin{tasks}
        \task R\$ 16,00.
        \task R\$ 24,00.
        \task R\$ 32,00.
        \task R\$ 40,00.
        \task R\$ 48,00.
    \end{tasks}
\end{question}

\begin{question}%[concurso:EAM; ano:2004; assunto:; alternativa:E]
Os irmãos Antônio e Pedro, sem nenhuma economia, receberam de seu pai uma certa quantia em dólares, cada um, para fazer uma viagem. Percebendo a diferença entre essas quantias, Antônio dá a Pedro tantos dólares quanto Antônio possui. Iniciam a viagem com US\$ 1 800,00 cada um. Quantos dólares cada um recebeu de seu pai inicialmente?
    \begin{tasks}
        \task Antônio recebeu US\$ 1 000,00 e Pedro US\$ 800,00.
        \task Antônio recebeu US\$ 2 000,00 e Pedro US\$ 2 250,00.
        \task Antônio recebeu US\$ 1 350,00 e Pedro US\$ 2 600,00.
        \task Antônio recebeu US\$ 2 250,00 e Pedro US\$ 1 000,00.
        \task Antônio recebeu US\$ 2 250,00 e Pedro US\$ 1 350,00.
    \end{tasks}
\end{question}

\begin{question}%[concurso:EAM; ano:2004; assunto:; alternativa:E]
O valor simplificado da expressão:
\[ \frac{1,363636 \ldots \times 2 \frac{1}{5} - (0,5)^2}{(\sqrt{2})^{-4}}\] é:
    \begin{tasks}
        \task \(\frac{9}{5}\).
        \task \(\frac{31}{5}\).
        \task \(7\).
        \task \(9\).
        \task \(11\).
    \end{tasks}
\end{question}

\begin{question}%[concurso:EAM; ano:2004; assunto:; alternativa:C]
Para monitorar duas avenidas, devem ser instaladas câmeras, posicionadas em pontos a partir da posição \(1\) até a posição \(n\) nas avenidas \(A\) e \(B\). Sendo \(u\) a maior e constante distância entre as câmeras, o total de câmeras a serem instaladas nas avenidas é:

INSERIR IMAGEM
    \begin{tasks}
        \task \(28\).
        \task \(30\).
        \task \(31\).
        \task \(36\).
        \task \(37\).
    \end{tasks}
\end{question}

\begin{question}%[concurso:EAM; ano:2004; assunto:; alternativa:B]
Para sustentação de letreiro é feito um suporte de ferro na forma de um triângulo retângulo \(ABC\). Calcule o comprimento de barra de ferro representada pelo segmento \(\overline{AD}\), sabendo que é bissetriz do ângulo \(B\hat{A}C\).

INSERIR IMAGEM
    \begin{tasks}
        \task \(0,56\)m.
        \task \(0,84\)m.
        \task \(0,92\)m.
        \task \(1\)m.
        \task \(1,2\)m.
    \end{tasks}
\end{question}

\begin{question}%[concurso:EAM; ano:2004; assunto:; alternativa:A]
Em uma viagem foram colocador dois tipos de revistas para que os tripulantes de um fragata desfrutassem de uma boa leitura. Ao final da viagem foi feita uma pesquisa com todos os tripulantes para saber das preferências com relação às revistas "saúde à bordo" ou "vid marinha", verificou-se que:

\begin{itemize}
    \item 20 tripulantes leram "saúde à bordo"
    \item 30 tripulantes leram "vida marinha"
    \item 8 tripulantes leram as duas revistas
    \item 14 tripulantes não leram nenhuma dessas revistas
\end{itemize}
Qual o número de tripulantes da fragata nesta viagem?
    \begin{tasks}
        \task \(56\).
        \task \(58\).
        \task \(64\).
        \task \(68\).
        \task \(72\).
    \end{tasks}
\end{question}

\begin{question}%[concurso:EAM; ano:2004; assunto:; alternativa:E]
Um marinheiro ao viajar comprou US\$ 1 000,00 a uma taxa de 2,9 Reais por Dólar. Não havendo usado este dinheiro na viagem, ele vendeu, na sua volta a uma taxa de 2,7 Reais por Dólar. Então:
    \begin{tasks}
        \task O marinheiro lucrou R\$ 180,00.
        \task O marinheiro lucrou R\$ 190,00.
        \task O marinheiro lucrou R\$ 200,00.
        \task O marinheiro perdeu R\$ 100,00.
        \task O marinheiro perdeu R\$ 200,00.
    \end{tasks}
\end{question}

\begin{question}%[concurso:EAM; ano:2004; assunto:; alternativa:D]
Numa competição de arremesso de dardo, o vencedor conseguiu 82m. O segundo colocado 78m. De quanto foi o lançamento do terceiro colocado, sabendo-se que a diferença entre seu lançamento e o lançamento do segundo colocado foi a terça parte da diferença entre o seu lançamento e o do primeiro?
    \begin{tasks}
        \task \(72\)m.
        \task \(74\)m.
        \task \(75\)m.
        \task \(76\)m.
        \task \(77\)m.
    \end{tasks}
\end{question}

\begin{question}%[concurso:EAM; ano:2004; assunto:; alternativa:D]
A somar das raízes reais da equação:
\[\sqrt{2} x^2 - (2\sqrt{2} + 2)x + 4 = 0\]
    \begin{tasks}
        \task \(0\).
        \task \(2 - \sqrt{2}\).
        \task \(\sqrt{2}\).
        \task \(2 + \sqrt{2}\).
        \task \(4\sqrt{2}\).
    \end{tasks}
\end{question}

\begin{question}%[concurso:EAM; ano:2004; assunto:; alternativa:C]
No numeral \(213a46\), a letra \(a\) representa um algarismo. Se o número correspondente é divisível por 3, a soma dos algarismos que podem substituir a letra \(a\) é:
    \begin{tasks}
        \task \(10\).
        \task \(12\).
        \task \(15\).
        \task \(16\).
        \task \(17\).
    \end{tasks}
\end{question}

% \begin{question}%[codigo:EAM200501AMX; concurso:EAM; ano:2005; assunto:; alternativa:]
Um cavalo deve ser amarrado a uma estaca situada em um dos vértices de um pasto que tem a forma de um quadrado, cujo lado mede 20m. Para que ele possa pastar em cerca de 20\% da área total do pasto, a parte inteira, em metros, do comprimento da corda que o prende á estaca deve ser igual a
    \begin{tasks}
        \task \(1\).
        \task \(2\).
        \task \(5\).
        \task \(8\).
        \task \(10\).
    \end{tasks}
\end{question}

\begin{question}%[codigo:EAM200502AMX; concurso:EAM; ano:2005; assunto:conjuntos, conjuntos numéricos; alternativa:]
Dado o seguinte problema: "Subtraindo-se \(3\) de um certo número \(x\), obtém-se o dobro da sua raiz quadrada. Qual é esse número?"; pode-se afirmar que, no conjunto dos números reais, esse problema
    \begin{tasks}
        \task tem duas soluções.
        \task tem só uma solução, a que é um número primo.
        \task tem só uma solução, a que é um número par.
        \task tem só uma solução, a que é um número ímpar e não primo.
        \task não tem solução.
    \end{tasks}
\end{question}

\begin{question}%[codigo:EAM200503AMX; concurso:EAM; ano:2005; assunto:geometria; alternativa:]

% \begin{figure}[h!]
    % 
    \includegraphics[width=.3\textwidth]{CONCURSO/EAM/IMAGES/2005/EAM200503IMG.png}
% \end{figure}


Na figura acima, \(AB = AC\), \(BX = BY\) e \(CZ = CY\). Se o ângulo \(A\) mede \(40^\circ\), quanto mede o ângulo \(XYZ\)?
    \begin{tasks}
        \task \(40^\circ\).
        \task \(50^\circ\).
        \task \(60^\circ\).
        \task \(70^\circ\).
        \task \(90^\circ\).
    \end{tasks}
\end{question}

\begin{question}%[codigo:EAM200504AMX; concurso:EAM; ano:2005; assunto:trigonometria no triângulo,trigonometria; alternativa:]

% \begin{figure}[h!]
    % 
    \includegraphics[width=.3\textwidth]{CONCURSO/EAM/IMAGES/2005/EAM200504IMG.png}
% \end{figure}

Uma escada de 10 metros de comprimento forma ângulo de \(60^\circ\) com a horizontal quando encostada ao edifício de um dos lados da rua, e ângulo de \(45^\circ\) se for encostada ao edifício do outro lado, apoiada no mesmo ponto do chão. A largura da rua, em metros, vale aproximadamente
    \begin{tasks}
        \task \(15\).
        \task \(14\).
        \task \(13\).
        \task \(12\).
        \task \(11\).
    \end{tasks}
\end{question}

\begin{question}%[codigo:EAM200505AMX; concurso:EAM; ano:2005; assunto:; alternativa:]
Uma balança assinala 325g para um certo copo cheio de água. Jogando-se metade da água fora, a balança passa a assinalar 180g. Para esse copo vazio, quanto tal balança assinalará em gramas?
    \begin{tasks}
        \task 20.
        \task 25.
        \task 35.
        \task 40.
        \task 45.
    \end{tasks}
\end{question}

\begin{question}%[codigo:EAM200506AMX; concurso:EAM; ano:2005; assunto:; alternativa:]
Numa competição de tiro-ao-alvo, cada atirador deve efetuar \(25\) disparos. Qual a porcentagem de acertos no alvo de um jogador que obtém \(+0,5\) pontos, sabendo-se que cada tiro no alvo vale \(+0,4\) e cada tiro fora do alvo vale \(-0,1\)?
    \begin{tasks}
        \task \(25\).
        \task \(24\).
        \task \(20\).
        \task \(16\).
        \task \(5\).
    \end{tasks}
\end{question}

\begin{question}%[codigo:EAM200507AMX; concurso:EAM; ano:2005; assunto:; alternativa:]
Um feirante compra duas unidades de maça por R\$ 0,75. Sabendo-se que ele vende o lote de seis maças por R\$ 3,00, quantas maças deverá vender para ter um lucro de R\$ 50,00?
    \begin{tasks}
        \task 50.
        \task 52.
        \task 400.
        \task 520.
        \task 600.
    \end{tasks}
\end{question}

\begin{question}%[codigo:EAM200508AMX; concurso:EAM; ano:2005; assunto:geometria,área de figuras; alternativa:]

% \begin{figure}[h!]
    
    \includegraphics[width=.5\textwidth]{CONCURSO/EAM/IMAGES/2005/EAM200508IMG.png}
% \end{figure}

Considerando-se que, nas figuras acima, os triângulos \(X,Y\) e \(Z\) estejam inscritos em retângulos congruentes, pode-se afirmar que
    \begin{tasks}
        \task apenas as áreas dos triângulos \(X\) e \(Y\) são iguais.
        \task apenas as áreas dos triângulos \(X\) e \(Y\) são iguais.
        \task apenas as áreas dos triângulos \(Y\) e \(Z\) são iguais.
        \task as áreas dos triângulos \(X,Y\) e \(Z\) são iguais entre si.
        \task as áreas dos triângulos \(X, Y\) e Z são diferentes entre si.
    \end{tasks}
\end{question}

\begin{question}%[codigo:EAM200509AMX; concurso:EAM; ano:2005; assunto:múltiplos; alternativa:]
Numa unidade da Marinha, estão lotados: 200 terceiros sargentos; 160 segundos sargentos; e \(n\) primeiros sargentos. Se \(n\) representa \(2/5\) do número total de sargentos da referida unidade, pode-se afirmar que \(n\)
    \begin{tasks}
        \task é múltiplo de 15 e de 8.
        \task é múltiplo de 15 e não de 8.
        \task não é múltiplo de 15, nem de 8.
        \task não é múltiplo de 15, mas é múltiplo de 8.
        \task é múltiplo de 18.
    \end{tasks}
\end{question}

\begin{question}%[codigo:EAM200510AMX; concurso:EAM; ano:2005; assunto:; alternativa:]
Em uma sala retangular de piso plano nas dimensões 8,80m por 7,60m, deseja-se colocar lajotas quadradas iguais sem a necessidade de recortar qualquer peça. A medida máximo, em centímetros, do lado de cada lajota deverá ser igual a
    \begin{tasks}
        \task 10.
        \task 20.
        \task 30.
        \task 40.
        \task 50.
    \end{tasks}
\end{question}

\begin{question}%[codigo:EAM200511AMX; concurso:EAM; ano:2005; assunto:; alternativa:]
Fatorando-se a expressão \(ac + 2bc - ad -2bd\), obtém-se
    \begin{tasks}
        \task \((a+2b)(c-d)\).
        \task \((a-2b)(c-d)\).
        \task \((a-2b)(c+d)\).
        \task \((a+c)^2(a-d)\).
        \task \((a-c)(a+2b)\).
    \end{tasks}
\end{question}

\begin{question}%[codigo:EAM200512AMX; concurso:EAM; ano:2005; assunto:; alternativa:]
Caso seja cobrado um imposto de 5\% sobre o valor de qualquer saque efetuado em uma instituição financeira, qual será o saque máximo possível, em reais, a ser efetuado em uma conta cujo saldo é de 2 100,00 reais?
    \begin{tasks}
        \task 1 995,00.
        \task 2 000,00.
        \task 2 050,00.
        \task 2 075,00.
        \task 2 095,00.
    \end{tasks}
\end{question}

\begin{question}%[codigo:EAM200513AMX; concurso:EAM; ano:2005; assunto:volume; alternativa:]
A maquete de um reservatório \(R\), feita na escala \(1:500\), tem 8mm de largura, 10mm de comprimento e 8mm de altura. Qual é a capacidade em litros do reservatório \(R\)?
    \begin{tasks}
        \task 640.
        \task 800.
        \task 6400.
        \task 8000.
        \task 80000.
    \end{tasks}
\end{question}

\begin{question}%[codigo:EAM200514AMX; concurso:EAM; ano:2005; assunto:trângulo; alternativa:]
Em um triângulo, os lados medem 9cm, 12cm e 15cm. Quanto mede, em centímetros, a altura relativa ao maior lado desse triângulo?
    \begin{tasks}
        \task 8,0.
        \task 7,2.
        \task 6,0.
        \task 5,6.
        \task 4,3.
    \end{tasks}
\end{question}

\begin{question}%[codigo:EAM200515AMX; concurso:EAM; ano:2005; assunto:perímetro; alternativa:]

% \begin{figure}[h!]
    
    \includegraphics[width=.5\textwidth]{CONCURSO/EAM/IMAGES/2005/EAM200515IMG.png}
% \end{figure}

Considerando-se que a figura \(A\) seja um retângulo e as figuras \(B\) e \(C\) sejam obtidas, respectivamente, pela retirada da figura \(A\) de um quadrado de lado unitário, pode-se afirmar que
    \begin{tasks}
        \task apenas os perímetros das figuras \(A\) e \(B\) são iguais.
        \task apenas os perímetros das figuras \(A\) e \(C\) são iguais.
        \task apenas os perímetros das figuras \(B\) e \(C\) são iguais.
        \task os perímetros das figuras \(A,B\) e \(C\) são todos iguais.
        \task os perímetros das figuras \(A,B\) e \(C\) são todos diferentes.
    \end{tasks}
\end{question}

% \input{CONCURSO/EAM/2006.tex}
% \begin{question}%[codigo:EAM200700AMX; concurso:EAM; ano:2007; assunto:; alternativa:]

    \begin{tasks}
        \task 
        \task 
        \task 
        \task 
        \task 
    \end{tasks}
\end{question}

\begin{question}%[codigo:EAM200700AMX; concurso:EAM; ano:2007; assunto:; alternativa:]

    \begin{tasks}
        \task 
        \task 
        \task 
        \task 
        \task 
    \end{tasks}
\end{question}

\begin{question}%[codigo:EAM200700AMX; concurso:EAM; ano:2007; assunto:; alternativa:]

    \begin{tasks}
        \task 
        \task 
        \task 
        \task 
        \task 
    \end{tasks}
\end{question}

\begin{question}%[codigo:EAM200700AMX; concurso:EAM; ano:2007; assunto:; alternativa:]

    \begin{tasks}
        \task 
        \task 
        \task 
        \task 
        \task 
    \end{tasks}
\end{question}

\begin{question}%[codigo:EAM200700AMX; concurso:EAM; ano:2007; assunto:; alternativa:]

    \begin{tasks}
        \task 
        \task 
        \task 
        \task 
        \task 
    \end{tasks}
\end{question}

\begin{question}%[codigo:EAM200700AMX; concurso:EAM; ano:2007; assunto:; alternativa:]

    \begin{tasks}
        \task 
        \task 
        \task 
        \task 
        \task 
    \end{tasks}
\end{question}

\begin{question}%[codigo:EAM200700AMX; concurso:EAM; ano:2007; assunto:; alternativa:]

    \begin{tasks}
        \task 
        \task 
        \task 
        \task 
        \task 
    \end{tasks}
\end{question}

\begin{question}%[codigo:EAM200700AMX; concurso:EAM; ano:2007; assunto:; alternativa:]

    \begin{tasks}
        \task 
        \task 
        \task 
        \task 
        \task 
    \end{tasks}
\end{question}

\begin{question}%[codigo:EAM200700AMX; concurso:EAM; ano:2007; assunto:; alternativa:]

    \begin{tasks}
        \task 
        \task 
        \task 
        \task 
        \task 
    \end{tasks}
\end{question}

\begin{question}%[codigo:EAM200700AMX; concurso:EAM; ano:2007; assunto:; alternativa:]

    \begin{tasks}
        \task 
        \task 
        \task 
        \task 
        \task 
    \end{tasks}
\end{question}

\begin{question}%[codigo:EAM200700AMX; concurso:EAM; ano:2007; assunto:; alternativa:]

    \begin{tasks}
        \task 
        \task 
        \task 
        \task 
        \task 
    \end{tasks}
\end{question}

\begin{question}%[codigo:EAM200700AMX; concurso:EAM; ano:2007; assunto:; alternativa:]

    \begin{tasks}
        \task 
        \task 
        \task 
        \task 
        \task 
    \end{tasks}
\end{question}

\begin{question}%[codigo:EAM200700AMX; concurso:EAM; ano:2007; assunto:; alternativa:]

    \begin{tasks}
        \task 
        \task 
        \task 
        \task 
        \task 
    \end{tasks}
\end{question}

\begin{question}%[codigo:EAM200700AMX; concurso:EAM; ano:2007; assunto:; alternativa:]

    \begin{tasks}
        \task 
        \task 
        \task 
        \task 
        \task 
    \end{tasks}
\end{question}

\begin{question}%[codigo:EAM200700AMX; concurso:EAM; ano:2007; assunto:; alternativa:]

    \begin{tasks}
        \task 
        \task 
        \task 
        \task 
        \task 
    \end{tasks}
\end{question}

% \begin{question}%[codigo:EAM200800AMX; concurso:EAM; ano:2008; assunto:; alternativa:]

    \begin{tasks}
        \task 
        \task 
        \task 
        \task 
        \task 
    \end{tasks}
\end{question}

\begin{question}%[codigo:EAM200800AMX; concurso:EAM; ano:2008; assunto:; alternativa:]

    \begin{tasks}
        \task 
        \task 
        \task 
        \task 
        \task 
    \end{tasks}
\end{question}

\begin{question}%[codigo:EAM200800AMX; concurso:EAM; ano:2008; assunto:; alternativa:]

    \begin{tasks}
        \task 
        \task 
        \task 
        \task 
        \task 
    \end{tasks}
\end{question}

\begin{question}%[codigo:EAM200800AMX; concurso:EAM; ano:2008; assunto:; alternativa:]

    \begin{tasks}
        \task 
        \task 
        \task 
        \task 
        \task 
    \end{tasks}
\end{question}

\begin{question}%[codigo:EAM200800AMX; concurso:EAM; ano:2008; assunto:; alternativa:]

    \begin{tasks}
        \task 
        \task 
        \task 
        \task 
        \task 
    \end{tasks}
\end{question}

\begin{question}%[codigo:EAM200800AMX; concurso:EAM; ano:2008; assunto:; alternativa:]

    \begin{tasks}
        \task 
        \task 
        \task 
        \task 
        \task 
    \end{tasks}
\end{question}

\begin{question}%[codigo:EAM200800AMX; concurso:EAM; ano:2008; assunto:; alternativa:]

    \begin{tasks}
        \task 
        \task 
        \task 
        \task 
        \task 
    \end{tasks}
\end{question}

\begin{question}%[codigo:EAM200800AMX; concurso:EAM; ano:2008; assunto:; alternativa:]

    \begin{tasks}
        \task 
        \task 
        \task 
        \task 
        \task 
    \end{tasks}
\end{question}

\begin{question}%[codigo:EAM200800AMX; concurso:EAM; ano:2008; assunto:; alternativa:]

    \begin{tasks}
        \task 
        \task 
        \task 
        \task 
        \task 
    \end{tasks}
\end{question}

\begin{question}%[codigo:EAM200800AMX; concurso:EAM; ano:2008; assunto:; alternativa:]

    \begin{tasks}
        \task 
        \task 
        \task 
        \task 
        \task 
    \end{tasks}
\end{question}

\begin{question}%[codigo:EAM200800AMX; concurso:EAM; ano:2008; assunto:; alternativa:]

    \begin{tasks}
        \task 
        \task 
        \task 
        \task 
        \task 
    \end{tasks}
\end{question}

\begin{question}%[codigo:EAM200800AMX; concurso:EAM; ano:2008; assunto:; alternativa:]

    \begin{tasks}
        \task 
        \task 
        \task 
        \task 
        \task 
    \end{tasks}
\end{question}

\begin{question}%[codigo:EAM200800AMX; concurso:EAM; ano:2008; assunto:; alternativa:]

    \begin{tasks}
        \task 
        \task 
        \task 
        \task 
        \task 
    \end{tasks}
\end{question}

\begin{question}%[codigo:EAM200800AMX; concurso:EAM; ano:2008; assunto:; alternativa:]

    \begin{tasks}
        \task 
        \task 
        \task 
        \task 
        \task 
    \end{tasks}
\end{question}

\begin{question}%[codigo:EAM200800AMX; concurso:EAM; ano:2008; assunto:; alternativa:]

    \begin{tasks}
        \task 
        \task 
        \task 
        \task 
        \task 
    \end{tasks}
\end{question}

% \begin{question}%[codigo:EAM200900AMX; concurso:EAM; ano:2009; assunto:; alternativa:]

    \begin{tasks}
        \task 
        \task 
        \task 
        \task 
        \task 
    \end{tasks}
\end{question}

\begin{question}%[codigo:EAM200900AMX; concurso:EAM; ano:2009; assunto:; alternativa:]

    \begin{tasks}
        \task 
        \task 
        \task 
        \task 
        \task 
    \end{tasks}
\end{question}

\begin{question}%[codigo:EAM200900AMX; concurso:EAM; ano:2009; assunto:; alternativa:]

    \begin{tasks}
        \task 
        \task 
        \task 
        \task 
        \task 
    \end{tasks}
\end{question}

\begin{question}%[codigo:EAM200900AMX; concurso:EAM; ano:2009; assunto:; alternativa:]

    \begin{tasks}
        \task 
        \task 
        \task 
        \task 
        \task 
    \end{tasks}
\end{question}

\begin{question}%[codigo:EAM200900AMX; concurso:EAM; ano:2009; assunto:; alternativa:]

    \begin{tasks}
        \task 
        \task 
        \task 
        \task 
        \task 
    \end{tasks}
\end{question}

\begin{question}%[codigo:EAM200900AMX; concurso:EAM; ano:2009; assunto:; alternativa:]

    \begin{tasks}
        \task 
        \task 
        \task 
        \task 
        \task 
    \end{tasks}
\end{question}

\begin{question}%[codigo:EAM200900AMX; concurso:EAM; ano:2009; assunto:; alternativa:]

    \begin{tasks}
        \task 
        \task 
        \task 
        \task 
        \task 
    \end{tasks}
\end{question}

\begin{question}%[codigo:EAM200900AMX; concurso:EAM; ano:2009; assunto:; alternativa:]

    \begin{tasks}
        \task 
        \task 
        \task 
        \task 
        \task 
    \end{tasks}
\end{question}

\begin{question}%[codigo:EAM200900AMX; concurso:EAM; ano:2009; assunto:; alternativa:]

    \begin{tasks}
        \task 
        \task 
        \task 
        \task 
        \task 
    \end{tasks}
\end{question}

\begin{question}%[codigo:EAM200900AMX; concurso:EAM; ano:2009; assunto:; alternativa:]

    \begin{tasks}
        \task 
        \task 
        \task 
        \task 
        \task 
    \end{tasks}
\end{question}

\begin{question}%[codigo:EAM200900AMX; concurso:EAM; ano:2009; assunto:; alternativa:]

    \begin{tasks}
        \task 
        \task 
        \task 
        \task 
        \task 
    \end{tasks}
\end{question}

\begin{question}%[codigo:EAM200900AMX; concurso:EAM; ano:2009; assunto:; alternativa:]

    \begin{tasks}
        \task 
        \task 
        \task 
        \task 
        \task 
    \end{tasks}
\end{question}

\begin{question}%[codigo:EAM200900AMX; concurso:EAM; ano:2009; assunto:; alternativa:]

    \begin{tasks}
        \task 
        \task 
        \task 
        \task 
        \task 
    \end{tasks}
\end{question}

\begin{question}%[codigo:EAM200900AMX; concurso:EAM; ano:2009; assunto:; alternativa:]

    \begin{tasks}
        \task 
        \task 
        \task 
        \task 
        \task 
    \end{tasks}
\end{question}

\begin{question}%[codigo:EAM200900AMX; concurso:EAM; ano:2009; assunto:; alternativa:]

    \begin{tasks}
        \task 
        \task 
        \task 
        \task 
        \task 
    \end{tasks}
\end{question}

% \begin{question}%[codigo:EAM201000AMX; concurso:EAM; ano:2010; assunto:; alternativa:]

    \begin{tasks}
        \task 
        \task 
        \task 
        \task 
        \task 
    \end{tasks}
\end{question}

\begin{question}%[codigo:EAM201000AMX; concurso:EAM; ano:2010; assunto:; alternativa:]

    \begin{tasks}
        \task 
        \task 
        \task 
        \task 
        \task 
    \end{tasks}
\end{question}

\begin{question}%[codigo:EAM201000AMX; concurso:EAM; ano:2010; assunto:; alternativa:]

    \begin{tasks}
        \task 
        \task 
        \task 
        \task 
        \task 
    \end{tasks}
\end{question}

\begin{question}%[codigo:EAM201000AMX; concurso:EAM; ano:2010; assunto:; alternativa:]

    \begin{tasks}
        \task 
        \task 
        \task 
        \task 
        \task 
    \end{tasks}
\end{question}

\begin{question}%[codigo:EAM201000AMX; concurso:EAM; ano:2010; assunto:; alternativa:]

    \begin{tasks}
        \task 
        \task 
        \task 
        \task 
        \task 
    \end{tasks}
\end{question}

\begin{question}%[codigo:EAM201000AMX; concurso:EAM; ano:2010; assunto:; alternativa:]

    \begin{tasks}
        \task 
        \task 
        \task 
        \task 
        \task 
    \end{tasks}
\end{question}

\begin{question}%[codigo:EAM201000AMX; concurso:EAM; ano:2010; assunto:; alternativa:]

    \begin{tasks}
        \task 
        \task 
        \task 
        \task 
        \task 
    \end{tasks}
\end{question}

\begin{question}%[codigo:EAM201000AMX; concurso:EAM; ano:2010; assunto:; alternativa:]

    \begin{tasks}
        \task 
        \task 
        \task 
        \task 
        \task 
    \end{tasks}
\end{question}

\begin{question}%[codigo:EAM201000AMX; concurso:EAM; ano:2010; assunto:; alternativa:]

    \begin{tasks}
        \task 
        \task 
        \task 
        \task 
        \task 
    \end{tasks}
\end{question}

\begin{question}%[codigo:EAM201000AMX; concurso:EAM; ano:2010; assunto:; alternativa:]

    \begin{tasks}
        \task 
        \task 
        \task 
        \task 
        \task 
    \end{tasks}
\end{question}

\begin{question}%[codigo:EAM201000AMX; concurso:EAM; ano:2010; assunto:; alternativa:]

    \begin{tasks}
        \task 
        \task 
        \task 
        \task 
        \task 
    \end{tasks}
\end{question}

\begin{question}%[codigo:EAM201000AMX; concurso:EAM; ano:2010; assunto:; alternativa:]

    \begin{tasks}
        \task 
        \task 
        \task 
        \task 
        \task 
    \end{tasks}
\end{question}

\begin{question}%[codigo:EAM201000AMX; concurso:EAM; ano:2010; assunto:; alternativa:]

    \begin{tasks}
        \task 
        \task 
        \task 
        \task 
        \task 
    \end{tasks}
\end{question}

\begin{question}%[codigo:EAM201000AMX; concurso:EAM; ano:2010; assunto:; alternativa:]

    \begin{tasks}
        \task 
        \task 
        \task 
        \task 
        \task 
    \end{tasks}
\end{question}

\begin{question}%[codigo:EAM201000AMX; concurso:EAM; ano:2010; assunto:; alternativa:]

    \begin{tasks}
        \task 
        \task 
        \task 
        \task 
        \task 
    \end{tasks}
\end{question}

% \begin{question}%[codigo:EAM201100AMX; concurso:EAM; ano:2011; assunto:; alternativa:]

    \begin{tasks}
        \task 
        \task 
        \task 
        \task 
        \task 
    \end{tasks}
\end{question}

\begin{question}%[codigo:EAM201100AMX; concurso:EAM; ano:2011; assunto:; alternativa:]

    \begin{tasks}
        \task 
        \task 
        \task 
        \task 
        \task 
    \end{tasks}
\end{question}

\begin{question}%[codigo:EAM201100AMX; concurso:EAM; ano:2011; assunto:; alternativa:]

    \begin{tasks}
        \task 
        \task 
        \task 
        \task 
        \task 
    \end{tasks}
\end{question}

\begin{question}%[codigo:EAM201100AMX; concurso:EAM; ano:2011; assunto:; alternativa:]

    \begin{tasks}
        \task 
        \task 
        \task 
        \task 
        \task 
    \end{tasks}
\end{question}

\begin{question}%[codigo:EAM201100AMX; concurso:EAM; ano:2011; assunto:; alternativa:]

    \begin{tasks}
        \task 
        \task 
        \task 
        \task 
        \task 
    \end{tasks}
\end{question}

\begin{question}%[codigo:EAM201100AMX; concurso:EAM; ano:2011; assunto:; alternativa:]

    \begin{tasks}
        \task 
        \task 
        \task 
        \task 
        \task 
    \end{tasks}
\end{question}

\begin{question}%[codigo:EAM201100AMX; concurso:EAM; ano:2011; assunto:; alternativa:]

    \begin{tasks}
        \task 
        \task 
        \task 
        \task 
        \task 
    \end{tasks}
\end{question}

\begin{question}%[codigo:EAM201100AMX; concurso:EAM; ano:2011; assunto:; alternativa:]

    \begin{tasks}
        \task 
        \task 
        \task 
        \task 
        \task 
    \end{tasks}
\end{question}

\begin{question}%[codigo:EAM201100AMX; concurso:EAM; ano:2011; assunto:; alternativa:]

    \begin{tasks}
        \task 
        \task 
        \task 
        \task 
        \task 
    \end{tasks}
\end{question}

\begin{question}%[codigo:EAM201100AMX; concurso:EAM; ano:2011; assunto:; alternativa:]

    \begin{tasks}
        \task 
        \task 
        \task 
        \task 
        \task 
    \end{tasks}
\end{question}

\begin{question}%[codigo:EAM201100AMX; concurso:EAM; ano:2011; assunto:; alternativa:]

    \begin{tasks}
        \task 
        \task 
        \task 
        \task 
        \task 
    \end{tasks}
\end{question}

\begin{question}%[codigo:EAM201100AMX; concurso:EAM; ano:2011; assunto:; alternativa:]

    \begin{tasks}
        \task 
        \task 
        \task 
        \task 
        \task 
    \end{tasks}
\end{question}

\begin{question}%[codigo:EAM201100AMX; concurso:EAM; ano:2011; assunto:; alternativa:]

    \begin{tasks}
        \task 
        \task 
        \task 
        \task 
        \task 
    \end{tasks}
\end{question}

\begin{question}%[codigo:EAM201100AMX; concurso:EAM; ano:2011; assunto:; alternativa:]

    \begin{tasks}
        \task 
        \task 
        \task 
        \task 
        \task 
    \end{tasks}
\end{question}

\begin{question}%[codigo:EAM201100AMX; concurso:EAM; ano:2011; assunto:; alternativa:]

    \begin{tasks}
        \task 
        \task 
        \task 
        \task 
        \task 
    \end{tasks}
\end{question}

% \begin{question}%[codigo:EAM201200AMX; concurso:EAM; ano:2012; assunto:; alternativa:]

    \begin{tasks}
        \task 
        \task 
        \task 
        \task 
        \task 
    \end{tasks}
\end{question}

\begin{question}%[codigo:EAM201200AMX; concurso:EAM; ano:2012; assunto:; alternativa:]

    \begin{tasks}
        \task 
        \task 
        \task 
        \task 
        \task 
    \end{tasks}
\end{question}

\begin{question}%[codigo:EAM201200AMX; concurso:EAM; ano:2012; assunto:; alternativa:]

    \begin{tasks}
        \task 
        \task 
        \task 
        \task 
        \task 
    \end{tasks}
\end{question}

\begin{question}%[codigo:EAM201200AMX; concurso:EAM; ano:2012; assunto:; alternativa:]

    \begin{tasks}
        \task 
        \task 
        \task 
        \task 
        \task 
    \end{tasks}
\end{question}

\begin{question}%[codigo:EAM201200AMX; concurso:EAM; ano:2012; assunto:; alternativa:]

    \begin{tasks}
        \task 
        \task 
        \task 
        \task 
        \task 
    \end{tasks}
\end{question}

\begin{question}%[codigo:EAM201200AMX; concurso:EAM; ano:2012; assunto:; alternativa:]

    \begin{tasks}
        \task 
        \task 
        \task 
        \task 
        \task 
    \end{tasks}
\end{question}

\begin{question}%[codigo:EAM201200AMX; concurso:EAM; ano:2012; assunto:; alternativa:]

    \begin{tasks}
        \task 
        \task 
        \task 
        \task 
        \task 
    \end{tasks}
\end{question}

\begin{question}%[codigo:EAM201200AMX; concurso:EAM; ano:2012; assunto:; alternativa:]

    \begin{tasks}
        \task 
        \task 
        \task 
        \task 
        \task 
    \end{tasks}
\end{question}

\begin{question}%[codigo:EAM201200AMX; concurso:EAM; ano:2012; assunto:; alternativa:]

    \begin{tasks}
        \task 
        \task 
        \task 
        \task 
        \task 
    \end{tasks}
\end{question}

\begin{question}%[codigo:EAM201200AMX; concurso:EAM; ano:2012; assunto:; alternativa:]

    \begin{tasks}
        \task 
        \task 
        \task 
        \task 
        \task 
    \end{tasks}
\end{question}

\begin{question}%[codigo:EAM201200AMX; concurso:EAM; ano:2012; assunto:; alternativa:]

    \begin{tasks}
        \task 
        \task 
        \task 
        \task 
        \task 
    \end{tasks}
\end{question}

\begin{question}%[codigo:EAM201200AMX; concurso:EAM; ano:2012; assunto:; alternativa:]

    \begin{tasks}
        \task 
        \task 
        \task 
        \task 
        \task 
    \end{tasks}
\end{question}

\begin{question}%[codigo:EAM201200AMX; concurso:EAM; ano:2012; assunto:; alternativa:]

    \begin{tasks}
        \task 
        \task 
        \task 
        \task 
        \task 
    \end{tasks}
\end{question}

\begin{question}%[codigo:EAM201200AMX; concurso:EAM; ano:2012; assunto:; alternativa:]

    \begin{tasks}
        \task 
        \task 
        \task 
        \task 
        \task 
    \end{tasks}
\end{question}

\begin{question}%[codigo:EAM201200AMX; concurso:EAM; ano:2012; assunto:; alternativa:]

    \begin{tasks}
        \task 
        \task 
        \task 
        \task 
        \task 
    \end{tasks}
\end{question}

% \begin{question}%[codigo:EAM201300AMX; concurso:EAM; ano:2013; assunto:; alternativa:]

    \begin{tasks}
        \task 
        \task 
        \task 
        \task 
        \task 
    \end{tasks}
\end{question}

\begin{question}%[codigo:EAM201300AMX; concurso:EAM; ano:2013; assunto:; alternativa:]

    \begin{tasks}
        \task 
        \task 
        \task 
        \task 
        \task 
    \end{tasks}
\end{question}

\begin{question}%[codigo:EAM201300AMX; concurso:EAM; ano:2013; assunto:; alternativa:]

    \begin{tasks}
        \task 
        \task 
        \task 
        \task 
        \task 
    \end{tasks}
\end{question}

\begin{question}%[codigo:EAM201300AMX; concurso:EAM; ano:2013; assunto:; alternativa:]

    \begin{tasks}
        \task 
        \task 
        \task 
        \task 
        \task 
    \end{tasks}
\end{question}

\begin{question}%[codigo:EAM201300AMX; concurso:EAM; ano:2013; assunto:; alternativa:]

    \begin{tasks}
        \task 
        \task 
        \task 
        \task 
        \task 
    \end{tasks}
\end{question}

\begin{question}%[codigo:EAM201300AMX; concurso:EAM; ano:2013; assunto:; alternativa:]

    \begin{tasks}
        \task 
        \task 
        \task 
        \task 
        \task 
    \end{tasks}
\end{question}

\begin{question}%[codigo:EAM201300AMX; concurso:EAM; ano:2013; assunto:; alternativa:]

    \begin{tasks}
        \task 
        \task 
        \task 
        \task 
        \task 
    \end{tasks}
\end{question}

\begin{question}%[codigo:EAM201300AMX; concurso:EAM; ano:2013; assunto:; alternativa:]

    \begin{tasks}
        \task 
        \task 
        \task 
        \task 
        \task 
    \end{tasks}
\end{question}

\begin{question}%[codigo:EAM201300AMX; concurso:EAM; ano:2013; assunto:; alternativa:]

    \begin{tasks}
        \task 
        \task 
        \task 
        \task 
        \task 
    \end{tasks}
\end{question}

\begin{question}%[codigo:EAM201300AMX; concurso:EAM; ano:2013; assunto:; alternativa:]

    \begin{tasks}
        \task 
        \task 
        \task 
        \task 
        \task 
    \end{tasks}
\end{question}

\begin{question}%[codigo:EAM201300AMX; concurso:EAM; ano:2013; assunto:; alternativa:]

    \begin{tasks}
        \task 
        \task 
        \task 
        \task 
        \task 
    \end{tasks}
\end{question}

\begin{question}%[codigo:EAM201300AMX; concurso:EAM; ano:2013; assunto:; alternativa:]

    \begin{tasks}
        \task 
        \task 
        \task 
        \task 
        \task 
    \end{tasks}
\end{question}

\begin{question}%[codigo:EAM201300AMX; concurso:EAM; ano:2013; assunto:; alternativa:]

    \begin{tasks}
        \task 
        \task 
        \task 
        \task 
        \task 
    \end{tasks}
\end{question}

\begin{question}%[codigo:EAM201300AMX; concurso:EAM; ano:2013; assunto:; alternativa:]

    \begin{tasks}
        \task 
        \task 
        \task 
        \task 
        \task 
    \end{tasks}
\end{question}

\begin{question}%[codigo:EAM201300AMX; concurso:EAM; ano:2013; assunto:; alternativa:]

    \begin{tasks}
        \task 
        \task 
        \task 
        \task 
        \task 
    \end{tasks}
\end{question}

% \begin{question}%[codigo:EAM201400AMX; concurso:EAM; ano:2014; assunto:; alternativa:]

    \begin{tasks}
        \task 
        \task 
        \task 
        \task 
        \task 
    \end{tasks}
\end{question}

\begin{question}%[codigo:EAM201400AMX; concurso:EAM; ano:2014; assunto:; alternativa:]

    \begin{tasks}
        \task 
        \task 
        \task 
        \task 
        \task 
    \end{tasks}
\end{question}

\begin{question}%[codigo:EAM201400AMX; concurso:EAM; ano:2014; assunto:; alternativa:]

    \begin{tasks}
        \task 
        \task 
        \task 
        \task 
        \task 
    \end{tasks}
\end{question}

\begin{question}%[codigo:EAM201400AMX; concurso:EAM; ano:2014; assunto:; alternativa:]

    \begin{tasks}
        \task 
        \task 
        \task 
        \task 
        \task 
    \end{tasks}
\end{question}

\begin{question}%[codigo:EAM201400AMX; concurso:EAM; ano:2014; assunto:; alternativa:]

    \begin{tasks}
        \task 
        \task 
        \task 
        \task 
        \task 
    \end{tasks}
\end{question}

\begin{question}%[codigo:EAM201400AMX; concurso:EAM; ano:2014; assunto:; alternativa:]

    \begin{tasks}
        \task 
        \task 
        \task 
        \task 
        \task 
    \end{tasks}
\end{question}

\begin{question}%[codigo:EAM201400AMX; concurso:EAM; ano:2014; assunto:; alternativa:]

    \begin{tasks}
        \task 
        \task 
        \task 
        \task 
        \task 
    \end{tasks}
\end{question}

\begin{question}%[codigo:EAM201400AMX; concurso:EAM; ano:2014; assunto:; alternativa:]

    \begin{tasks}
        \task 
        \task 
        \task 
        \task 
        \task 
    \end{tasks}
\end{question}

\begin{question}%[codigo:EAM201400AMX; concurso:EAM; ano:2014; assunto:; alternativa:]

    \begin{tasks}
        \task 
        \task 
        \task 
        \task 
        \task 
    \end{tasks}
\end{question}

\begin{question}%[codigo:EAM201400AMX; concurso:EAM; ano:2014; assunto:; alternativa:]

    \begin{tasks}
        \task 
        \task 
        \task 
        \task 
        \task 
    \end{tasks}
\end{question}

\begin{question}%[codigo:EAM201400AMX; concurso:EAM; ano:2014; assunto:; alternativa:]

    \begin{tasks}
        \task 
        \task 
        \task 
        \task 
        \task 
    \end{tasks}
\end{question}

\begin{question}%[codigo:EAM201400AMX; concurso:EAM; ano:2014; assunto:; alternativa:]

    \begin{tasks}
        \task 
        \task 
        \task 
        \task 
        \task 
    \end{tasks}
\end{question}

\begin{question}%[codigo:EAM201400AMX; concurso:EAM; ano:2014; assunto:; alternativa:]

    \begin{tasks}
        \task 
        \task 
        \task 
        \task 
        \task 
    \end{tasks}
\end{question}

\begin{question}%[codigo:EAM201400AMX; concurso:EAM; ano:2014; assunto:; alternativa:]

    \begin{tasks}
        \task 
        \task 
        \task 
        \task 
        \task 
    \end{tasks}
\end{question}

\begin{question}%[codigo:EAM201400AMX; concurso:EAM; ano:2014; assunto:; alternativa:]

    \begin{tasks}
        \task 
        \task 
        \task 
        \task 
        \task 
    \end{tasks}
\end{question}

% \begin{question}%[codigo:EAM201500AMX; concurso:EAM; ano:2015; assunto:; alternativa:]

    \begin{tasks}
        \task 
        \task 
        \task 
        \task 
        \task 
    \end{tasks}
\end{question}

\begin{question}%[codigo:EAM201500AMX; concurso:EAM; ano:2015; assunto:; alternativa:]

    \begin{tasks}
        \task 
        \task 
        \task 
        \task 
        \task 
    \end{tasks}
\end{question}

\begin{question}%[codigo:EAM201500AMX; concurso:EAM; ano:2015; assunto:; alternativa:]

    \begin{tasks}
        \task 
        \task 
        \task 
        \task 
        \task 
    \end{tasks}
\end{question}

\begin{question}%[codigo:EAM201500AMX; concurso:EAM; ano:2015; assunto:; alternativa:]

    \begin{tasks}
        \task 
        \task 
        \task 
        \task 
        \task 
    \end{tasks}
\end{question}

\begin{question}%[codigo:EAM201500AMX; concurso:EAM; ano:2015; assunto:; alternativa:]

    \begin{tasks}
        \task 
        \task 
        \task 
        \task 
        \task 
    \end{tasks}
\end{question}

\begin{question}%[codigo:EAM201500AMX; concurso:EAM; ano:2015; assunto:; alternativa:]

    \begin{tasks}
        \task 
        \task 
        \task 
        \task 
        \task 
    \end{tasks}
\end{question}

\begin{question}%[codigo:EAM201500AMX; concurso:EAM; ano:2015; assunto:; alternativa:]

    \begin{tasks}
        \task 
        \task 
        \task 
        \task 
        \task 
    \end{tasks}
\end{question}

\begin{question}%[codigo:EAM201500AMX; concurso:EAM; ano:2015; assunto:; alternativa:]

    \begin{tasks}
        \task 
        \task 
        \task 
        \task 
        \task 
    \end{tasks}
\end{question}

\begin{question}%[codigo:EAM201500AMX; concurso:EAM; ano:2015; assunto:; alternativa:]

    \begin{tasks}
        \task 
        \task 
        \task 
        \task 
        \task 
    \end{tasks}
\end{question}

\begin{question}%[codigo:EAM201500AMX; concurso:EAM; ano:2015; assunto:; alternativa:]

    \begin{tasks}
        \task 
        \task 
        \task 
        \task 
        \task 
    \end{tasks}
\end{question}

\begin{question}%[codigo:EAM201500AMX; concurso:EAM; ano:2015; assunto:; alternativa:]

    \begin{tasks}
        \task 
        \task 
        \task 
        \task 
        \task 
    \end{tasks}
\end{question}

\begin{question}%[codigo:EAM201500AMX; concurso:EAM; ano:2015; assunto:; alternativa:]

    \begin{tasks}
        \task 
        \task 
        \task 
        \task 
        \task 
    \end{tasks}
\end{question}

\begin{question}%[codigo:EAM201500AMX; concurso:EAM; ano:2015; assunto:; alternativa:]

    \begin{tasks}
        \task 
        \task 
        \task 
        \task 
        \task 
    \end{tasks}
\end{question}

\begin{question}%[codigo:EAM201500AMX; concurso:EAM; ano:2015; assunto:; alternativa:]

    \begin{tasks}
        \task 
        \task 
        \task 
        \task 
        \task 
    \end{tasks}
\end{question}

\begin{question}%[codigo:EAM201500AMX; concurso:EAM; ano:2015; assunto:; alternativa:]

    \begin{tasks}
        \task 
        \task 
        \task 
        \task 
        \task 
    \end{tasks}
\end{question}

% \begin{question}%[codigo:%[codigo:EAM202201AMX; concurso:EAM;AMX; concurso:EAM; ano:2016; assunto:; alternativa:]

    \begin{tasks}
        \task 
        \task 
        \task 
        \task 
        \task 
    \end{tasks}
\end{question}

\begin{question}%[codigo:EAM201600AMX; concurso:EAM; ano:2016; assunto:; alternativa:]

    \begin{tasks}
        \task 
        \task 
        \task 
        \task 
        \task 
    \end{tasks}
\end{question}

\begin{question}%[codigo:EAM201600AMX; concurso:EAM; ano:2016; assunto:; alternativa:]

    \begin{tasks}
        \task 
        \task 
        \task 
        \task 
        \task 
    \end{tasks}
\end{question}

\begin{question}%[codigo:EAM201600AMX; concurso:EAM; ano:2016; assunto:; alternativa:]

    \begin{tasks}
        \task 
        \task 
        \task 
        \task 
        \task 
    \end{tasks}
\end{question}

\begin{question}%[codigo:EAM201600AMX; concurso:EAM; ano:2016; assunto:; alternativa:]

    \begin{tasks}
        \task 
        \task 
        \task 
        \task 
        \task 
    \end{tasks}
\end{question}

\begin{question}%[codigo:EAM201600AMX; concurso:EAM; ano:2016; assunto:; alternativa:]

    \begin{tasks}
        \task 
        \task 
        \task 
        \task 
        \task 
    \end{tasks}
\end{question}

\begin{question}%[codigo:EAM201600AMX; concurso:EAM; ano:2016; assunto:; alternativa:]

    \begin{tasks}
        \task 
        \task 
        \task 
        \task 
        \task 
    \end{tasks}
\end{question}

\begin{question}%[codigo:EAM201600AMX; concurso:EAM; ano:2016; assunto:; alternativa:]

    \begin{tasks}
        \task 
        \task 
        \task 
        \task 
        \task 
    \end{tasks}
\end{question}

\begin{question}%[codigo:EAM201600AMX; concurso:EAM; ano:2016; assunto:; alternativa:]

    \begin{tasks}
        \task 
        \task 
        \task 
        \task 
        \task 
    \end{tasks}
\end{question}

\begin{question}%[codigo:EAM201600AMX; concurso:EAM; ano:2016; assunto:; alternativa:]

    \begin{tasks}
        \task 
        \task 
        \task 
        \task 
        \task 
    \end{tasks}
\end{question}

\begin{question}%[codigo:EAM201600AMX; concurso:EAM; ano:2016; assunto:; alternativa:]

    \begin{tasks}
        \task 
        \task 
        \task 
        \task 
        \task 
    \end{tasks}
\end{question}

\begin{question}%[codigo:EAM201600AMX; concurso:EAM; ano:2016; assunto:; alternativa:]

    \begin{tasks}
        \task 
        \task 
        \task 
        \task 
        \task 
    \end{tasks}
\end{question}

\begin{question}%[codigo:EAM201600AMX; concurso:EAM; ano:2016; assunto:; alternativa:]

    \begin{tasks}
        \task 
        \task 
        \task 
        \task 
        \task 
    \end{tasks}
\end{question}

\begin{question}%[codigo:EAM201600AMX; concurso:EAM; ano:2016; assunto:; alternativa:]

    \begin{tasks}
        \task 
        \task 
        \task 
        \task 
        \task 
    \end{tasks}
\end{question}

\begin{question}%[codigo:EAM201600AMX; concurso:EAM; ano:2016; assunto:; alternativa:]

    \begin{tasks}
        \task 
        \task 
        \task 
        \task 
        \task 
    \end{tasks}
\end{question}

% \begin{question}%[concurso:EAM; ano:2017; assunto:; alternativa:]
Sendo \(x - \frac{2}{x} = a\), então \(x^2 + \frac{4}{x}\) é igual a:
    \begin{tasks}
        \task \(a^2 + 4\).
        \task \(a^2 - 4\).
        \task \(a^2\).
        \task \(a + 4\).
        \task \(a - 4\).
    \end{tasks}
\end{question}

\begin{question}%[concurso:EAM; ano:2017; assunto:; alternativa:]
Analise a figura a seguir.

INSERIR FIGURA

Calcule a soma das áreas hachuradas da figura acima, sabendo que os polígonos I e II são quadrados, e assinale a opção correta.
    \begin{tasks}
        \task \(22\sqrt{3}\).
        \task \(22\).
        \task \(13 + 4\sqrt{3}\).
        \task \(11\).
        \task \(11\sqrt{3}\).
    \end{tasks}
\end{question}

\begin{question}%[concurso:EAM; ano:2017; assunto:; alternativa:]
Observe a figura a seguir.

INSERIR FIGURA

Sabendo que, na figura acima, as reatas \(r\) e \(s\) são paralelas, é correto afirmar que o valor de \(x\) é igual a:
    \begin{tasks}
        \task 90\(^\circ\).
        \task 85\(^\circ\).
        \task 80\(^\circ\).
        \task 75\(^\circ\).
        \task 70\(^\circ\).
    \end{tasks}
\end{question}

\begin{question}%[concurso:EAM; ano:2017; assunto:; alternativa:]
Deseja-se azulejar, até o teto, as 4 paredes de uma cozinha. Sabe-se que a cozinha possui 2 portas medindo 210cm de altura e 80cm de largura cada uma, e uma janela com 150cm de altura e 110cm de comprimento. O comprimento, a largura e a altura da cozinha são iguais a 5,0m, 4,0m e 3,0m, respectivamente. Determine o número mínimo de metros quadrados inteiros de azulejos que devem ser comprados a assinale a opção correta.
    \begin{tasks}
        \task 42.
        \task 43.
        \task 49.
        \task 55.
        \task 58.
    \end{tasks}
\end{question}

\begin{question}%[concurso:EAM; ano:2017; assunto:; alternativa:]
Considerando \(n(P)\) como a notação que determina o número de elementos de um conjunto \(P,A \times X\) como o produto cartesiano entre dois conjuntos finitos A e B e sabendo-se ainda que \(n(A) = 2x - 3, n(B) = x-5\) e \(n(A \times B)= x^2 +10x -27\), é correto afirmar que o valor numérico de \(x\) é:
    \begin{tasks}
        \task um número primo.
        \task um múltiplo de 5.
        \task um múltiplo de 7.
        \task um múltiplo de 11.
        \task um múltiplo de 13.
    \end{tasks}
\end{question}

\begin{question}%[concurso:EAM; ano:2017; assunto:; alternativa:]
Seja a função real \(f\) definida por \(f(x) = \frac{x+k}{p}\). Sabendo-se que \(f(3) = 2\) e \(f(5) = 4\), determine o valor de \(k+p\) e assinale a opção correta.
    \begin{tasks}
        \task 0.
        \task 1.
        \task 2.
        \task 3.
        \task 4.
    \end{tasks}
\end{question}

\begin{question}%[concurso:EAM; ano:2017; assunto:; alternativa:]
Sabendo-se que \(A\) e \(B\) são subconjuntos finitos de \(U\), que \( \bar{A}\) é a notação para a operação complementar de \(A\) em relação a \(U\), que \(\bar{A} = \{q,r,s,t,u\}, A \cap B = \{o,p\}\) e \(A \cup B = \{m,n,o,p,q,r\}\), é correto afirmar que:
    \begin{tasks}
        \task \(A\) tem dois elementos e \(B\) tem quatro elementos.
        \task \(A\) tem quatro elementos e \(B\) tem dois elementos.
        \task \(A\) tem três elementos e \(B\) tem três elementos.
        \task \(A\) tem quatro elementos e \(B\) tem quatro elementos.
        \task \(A\) tem um elemento e \(B\) tem cinco elementos.
    \end{tasks}
\end{question}

\begin{question}%[concurso:EAM; ano:2017; assunto:; alternativa:]
Sabendo que a fração \(\frac{y}{4}\) é proporcional à fração \(\frac{3}{6-2\sqrt{3}}\), é correto afirmar que \(y\) é igual a:
    \begin{tasks}
        \task \( 1 - 2\sqrt{3}\).
        \task \( 6 + 3\sqrt{3}\).
        \task \( 2 - \sqrt{3}\).
        \task \( 4 + 3\sqrt{3}\).
        \task \( 3+ \sqrt{3}\).
    \end{tasks}
\end{question}

\begin{question}%[concurso:EAM; ano:2017; assunto:; alternativa:]
A soma de um número \(x\) com o dobro de um número \(y\) é \(-7\); e a diferença entre o triplo desse número \(x\) e o número \(y\) é igual a \(7\). Sendo assim, é correto afirmar que o produto \(xy\) é igual a:
    \begin{tasks}
        \task -15.
        \task -12.
        \task -10.
        \task -4.
        \task -2.
    \end{tasks}
\end{question}

\begin{question}%[concurso:EAM; ano:2017; assunto:; alternativa:]
O número natural \(N= 2^3 \cdot 3^p\) possui 20 divisores positivos. Sendo assim, o valor de \(p\) é:
    \begin{tasks}
        \task 2.
        \task 3.
        \task 4.
        \task 5.
        \task 6.
    \end{tasks}
\end{question}

\begin{question}%[concurso:EAM; ano:2017; assunto:; alternativa:]
Apoiado em dois pilares construídos sobre um terreno plano e distantes 3m um do outro, constrói-se um telhado, cuja inclinação é de 30\(^\circ\) em relação ao piso. Se o pular de menor altura mede 4 metros, qual é a altura do outro pilar? Dado: \(\sqrt{3} = 1,7\)
    \begin{tasks}
        \task 5,5m.
        \task 5,7m.
        \task 6,0m.
        \task 6,5m.
        \task 6,9m.
    \end{tasks}
\end{question}

\begin{question}%[concurso:EAM; ano:2017; assunto:; alternativa:]
Um colecionador de selos criou um catálogo de selos em uma pasta com 20 páginas numeradas de 1 até 20, cada uma com 15 selos, distribuídos em 5 linhas e 3 colunas. Os selos foram numerados de 1 a 300. Nesse catálogo, alguns selos são considerados raros e ocupam as posições 9\textordfeminine{},18\textordfeminine{},27\textordfeminine{},36\textordfeminine{} e assim sucessivamente. Depois que o catálogo for completado com todos os selos, é correto afirmar que o número de última página que terminará com um selo raro será
    \begin{tasks}
        \task 9.
        \task 11.
        \task 12.
        \task 18.
        \task 20.
    \end{tasks}
\end{question}

\begin{question}%[concurso:EAM; ano:2017; assunto:; alternativa:]
No dia 17-10-2016, á zero hora, iniciou-se mais uma vez o horário de verão no Rio de Janeiro, que tem sido usado com objetivo de economizar energia elétrica nos momentos de pico e evitar sobrecarga no sistema. No dia 16-10-2016, um avião partiu de St. John's, Canadá, com destino ao Rio de Janeiro. A saída aconteceu às 21h e 45min e o voo teve duração de 13h e 45min. Considerando que entre St. John's e Rio de Janeiro não há diferença de fuso horário, a que horas local o avião chegou ao Rio de Janeiro?
    \begin{tasks}
        \task 9h e 30min.
        \task 10h e 30min.
        \task 11h e 15min.
        \task 11h e 45min.
        \task 12h e 30min.
    \end{tasks}
\end{question}

\begin{question}%[concurso:EAM; ano:2017; assunto:; alternativa:]
Observe a figura a seguir.

INSERIR IMAGEM

Na figura acima, tem-se um triângulo isósceles \(ACD\), no qual o segmento \(\overline{AB}\) mede 3cm, o lado desigual \(AD\) mede \(10\sqrt{2}\)cm e os segmentos \(\overline{AC}\) e \(\overline{CD}\) são perpendiculares. Sendo assim, é correto afirmar que o segmento \(\overline{BD}\) mede:
    \begin{tasks}
        \task \(\sqrt{53}\)cm.
        \task \(\sqrt{97}\)cm.
        \task \(\sqrt{111}\)cm.
        \task \(\sqrt{149}\)cm.
        \task \(\sqrt{161}\)cm.
    \end{tasks}
\end{question}

\begin{question}%[concurso:EAM; ano:2017; assunto:; alternativa:]
A área de um retângulo corresponde à expressão \(K^2-10k-24\) quando \(k=36\). Sendo assim, calcule suas dimensões e assinale a opção correta.
    \begin{tasks}
        \task 38 e 24.
        \task 36 e 32.
        \task 63 e 24.
        \task 54 e 38.
        \task 32 e 24.
    \end{tasks}
\end{question}

\begin{question}%[concurso:EAM; ano:2017; assunto:; alternativa:]
Observe a figura abaixo.

INSERIR FIGURA

Um prédio projeta no solo uma sombra de 30m de extensão no mesmo instante em que uma pessoa de 1,80m projeta uma sombra de 2,0m. Pode-se afirmar que a altura do prédio vale
    \begin{tasks}
        \task 27m.
        \task 30m.
        \task 33m.
        \task 36m.
        \task 40m.
    \end{tasks}
\end{question}
% \begin{question}%[concurso:EAM; ano:2018; assunto:; alternativa:]
A partir de um dos vértices de um polígono convexo pode-se traçar tantas diagonais quantas são o total de diagonais de um pentágono. É correto afirmar que esse polígono é um:
    \begin{tasks}
        \task Hexágono.
        \task Heptágono.
        \task Octógono.
        \task Decágono.
        \task Dodecágono.
    \end{tasks}
\end{question}

\begin{question}%[concurso:EAM; ano:2018; assunto:; alternativa:]
Considere a função \(f(x) = k \cos(x)\), onde \(K\) é uma constante real, diferente de zero, e \(x\) é valor de graus. É correto afirmar que a razão entre \(f(60^\circ)\) e \(f(45^\circ)\) é igual a:
    \begin{tasks}
        \task \(\frac{\sqrt{2}}{2}\).
        \task \(\frac{1}{2}\).
        \task \(\frac{\sqrt{3}}{2}\).
        \task \(\frac{\sqrt{2}}{3}\).
        \task \(2\).
    \end{tasks}
\end{question}

\begin{question}%[concurso:EAM; ano:2018; assunto:; alternativa:]
Observe a figura abaixo.

INSERIR FIGURA

Uma piscina se utiliza das duas torneiras e do ralo da figura acima para manutenção do seu nível de água. A torneira \(B\), aberta sozinha, enche a piscina em 6 horas e torneira \(A\), também sozinha, enche a piscina é 4 horas. Caso a piscina esteja cheia, o ralo esvaziará num tempo \(t\). Num certo dia, o piscineiro, estando a piscina vazia, abriu as duas torneiras, porém esqueceu de fechar o ralo constatando posteriormente que a piscina ficou completamente cheia, nessas condições, em 12 horas. Sendo assim, é correto afirmar que essa piscina com as duas torneiras fechadas e o ralo aberto, estando totalmente cheia, necessitará de \(t\) horas para esvaziá-la, sendo \(t\) igual a:
    \begin{tasks}
        \task 3.
        \task 5.
        \task 7.
        \task 9.
        \task 12.
    \end{tasks}
\end{question}

\begin{question}%[concurso:EAM; ano:2018; assunto:; alternativa:]
É correto afirmar que o valor da soma das raízes reais da equação \(x^4 = 7x^2 + 18\) é um número:
    \begin{tasks}
        \task primo.
        \task divisor de 36.
        \task múltiplo de 3.
        \task divisor de 16.
        \task divisor de 25.
    \end{tasks}
\end{question}

\begin{question}%[concurso:EAM; ano:2018; assunto:; alternativa:]
Se a soma dos quadrados das raízes da equação \(x^2 + px + 10 = 0\) é igual a \(29\), é correto afirmar que o valor de \(p^2\) é um múltiplo de:
    \begin{tasks}
        \task 2.
        \task 3.
        \task 5.
        \task 7.
        \task 9.
    \end{tasks}
\end{question}

\begin{question}%[concurso:EAM; ano:2018; assunto:; alternativa:]
Analise a figura a seguir.

INSERIR FIGURA

Na figura, \(AB = AC, BX=BY\) e \(CZ=CY\). Se o ângulo \(A\) mede 40\(^\circ\), então o ângulo \(XYZ\) mede:
    \begin{tasks}
        \task 40\(^\circ\).
        \task 50\(^\circ\).
        \task 60\(^\circ\).
        \task 70\(^\circ\).
        \task 90\(^\circ\).
    \end{tasks}
\end{question}

\begin{question}%[concurso:EAM; ano:2018; assunto:; alternativa:]
Analise a figura abaixo.

INSERIR IMAGEM

A área do trapézio da figura acima é 12. Considere que o segmento \(EC = 4; CD = 2\) e \(GH = 2r\). Considere, ainda, que os pontos \(C,G\) e \(H\) são pontos de tangência e \(r\) é o raio do semicírculo sombreado. Sendo assim, é correto afirmar que a área do semicírculo sombreado é igual a:
    \begin{tasks}
        \task \(\pi\)
        \task \(2\pi\)
        \task \(3\pi\)
        \task \(4\pi\)
        \task \(5\pi\)
    \end{tasks}
\end{question}

\begin{question}%[concurso:EAM; ano:2018; assunto:; alternativa:]
Analise a figura a seguir.

INSERIR FIGURA

Um arquiteto pretende fixar em um painel de 40m de comprimento horizontal sete gravuras com 4m de comprimento horizontal cada. a distância entre duas gravuras consecutivas é \(d\), enquanto que a distância da primeira e da última gravura até as respectivas laterais do painel é \(2d\). Sendo assim, é correto afirmar que \(d\) é igual a:
    \begin{tasks}
        \task 0,85m.
        \task 1,15m.
        \task 1,20m.
        \task 1,25m.
        \task 1,35m.
    \end{tasks}
\end{question}

\begin{question}%[concurso:EAM; ano:2018; assunto:; alternativa:]
Analise as afirmativas abaixo:

\begin{enumerate}[label=\Roman*.]
    \item Todo quadrado é um losango.
    \item Todo quadrado é um retângulo.
    \item Todo retângulo é um paralelogramo.
    \item Todo triângulo equilátero é isósceles.
\end{enumerate}

Assinale a opção correta.
    \begin{tasks}
        \task Apenas a afirmativa I é verdadeira.
        \task As afirmativas I,II,III e IV são verdadeiras.
        \task Apenas as afirmativas I, II e III são verdadeiras.
        \task Apenas as afirmativas III e IV são verdadeiras.
        \task Apenas a afirmativa II é verdadeira.
    \end{tasks}
\end{question}

\begin{question}%[concurso:EAM; ano:2018; assunto:; alternativa:]
A expressão \(\frac{\frac{x}{2x-1}-1}{1+ \frac{x}{1-2x}}\) para \(x \neq 1, x \neq \frac{1}{2}\) e \(x \neq -\frac{1}{2}\) é igual a:
    \begin{tasks}
        \task -2.
        \task -1.
        \task 0.
        \task 2.
        \task 3.
    \end{tasks}
\end{question}

\begin{question}%[concurso:EAM; ano:2018; assunto:; alternativa:]
Se \(A=\sqrt{\sqrt{6}-2} \cdot \sqrt{2+\sqrt{6}}\), então o valor de \(A^2\) é:
    \begin{tasks}
        \task 1.
        \task 2.
        \task 4.
        \task 6.
        \task 36.
    \end{tasks}
\end{question}

\begin{question}%[concurso:EAM; ano:2018; assunto:; alternativa:]
Uma padaria produz 800 pães e, para essa produção, necessita de 12 litros de leite. Se a necessidade de leite é proporcional à produção, se o dono quer aumentar a produção de pães em 25\% e se o litro de leite custa R\$ 2,50, quanto o dono deverá gastar a mais com a compra de leite para atingir sua meta?
    \begin{tasks}
        \task R\$ 5,00.
        \task R\$ 7,50.
        \task R\$ 20,00.
        \task R\$ 30,00.
        \task R\$ 37,50.
    \end{tasks}
\end{question}

\begin{question}%[concurso:EAM; ano:2018; assunto:; alternativa:]
Sabendo-se que \(x - \frac{1}{x} = 1\) é correto afirmar que \(x^3 - \frac{1}{x^3}\) é igual a:
    \begin{tasks}
        \task 1.
        \task 4.
        \task 8.
        \task 12.
        \task 27.
    \end{tasks}
\end{question}

\begin{question}%[concurso:EAM; ano:2018; assunto:; alternativa:]
Dentre os inscritos em um concurso público, 60\% são homens e 40\% são mulheres. Sabe-se que já estão empregados 80\% dos homens e 30\% das mulheres. Qual a porcentagem dos candidatos que já tem emprego?
    \begin{tasks}
        \task 60\%.
        \task 40\%.
        \task 30\%.
        \task 24\%.
        \task 12\%.
    \end{tasks}
\end{question}

\begin{question}%[concurso:EAM; ano:2018; assunto:; alternativa:]
Considerando-se todos os divisores naturais de 360, quantos NÃO são pares?
    \begin{tasks}
        \task 6.
        \task 5.
        \task 4.
        \task 3.
        \task 2.
    \end{tasks}
\end{question}
% \begin{question}%[codigo:EAM201901AMX; concurso:EAM; ano:2019; assunto:; alternativa:]
Seja \(f\) uma função real, definida por \(f(x) = x^2 - 3x +2\). O conjunto imagem dessa função é o intervalo:
    \begin{tasks}
        \task \(\left[ -\frac{1}{3}; + \infty \right)\).
        \task \(\left[ -\frac{1}{6}; + \infty \right)\).
        \task \(\left[ -\frac{1}{4}; + \infty \right)\).
        \task \(\left[ -\frac{1}{2}; + \infty \right)\).
        \task \(\left[ \frac{1}{4}; + \infty \right)\).
    \end{tasks}
\end{question}

\begin{question}%[codigo:EAM201902AMX; concurso:EAM; ano:2019; assunto:; alternativa:]
A expressão \(\frac{2+a^2-3a}{6+a^2-5a} \div \frac{4+a^2-5a}{12-7a+a^2}\), quando simplificada, considerando a condição de existência dessa simplificação, tem como resultado:
    \begin{tasks}
        \task \( a^2 + 1\).
        \task \( a+1\).
        \task \( 2\).
        \task \( 1\).
        \task \( a-1\).
    \end{tasks}
\end{question}

\begin{question}%[codigo:EAM201903AMX; concurso:EAM; ano:2019; assunto:; alternativa:]
Sendo um hexágono regular inscrito em um círculo de raio 2, calcule a medida da diagonal maior desse hexágono e assinale a opção correta.
    \begin{tasks}
        \task \(4\).
        \task \(4\sqrt{3}\).
        \task \(8\).
        \task \(6\sqrt{3}\).
        \task \(12\).
    \end{tasks}
\end{question}

\begin{question}%[codigo:EAM201904AMX; concurso:EAM; ano:2019; assunto:; alternativa:]
Para vender seus produtos, um comerciante reduziu os preços dos brinquedos em 10\%. Depois que houve uma recuperação nas vendas, decidiu restaurar o valor antigo. Sendo assim, o novo preço deve ser aumentado aproximadamente em:
    \begin{tasks}
        \task 9\%.
        \task 11\%.
        \task 13\%.
        \task 15\%.
        \task 17\%.
    \end{tasks}
\end{question}

\begin{question}%[codigo:EAM201905AMX; concurso:EAM; ano:2019; assunto:; alternativa:]
O conjunto solução, nos reais, na inequação \(\frac{5}{x-1}> 1\) é o intervalo:
    \begin{tasks}
        \task \(]5,6[\).
        \task \(]-\infty,6[\).
        \task \(\mathbb{R}\).
        \task \(]1, +\infty[\).
        \task \(]1,6[\).
    \end{tasks}
\end{question}

\begin{question}%[codigo:EAM201906AMX; concurso:EAM; ano:2019; assunto:; alternativa:]
Sendo \(x\) real tal que \(\sen x = \frac{m-1}{2}\) e \(\cos x = \frac{m+1}{2}\). Determine o conjunto dos valores de \(m\) e assinale a opção correta.
    \begin{tasks}
        \task \(\{-\sqrt{2},\sqrt{2}\}\)
        \task \(\{-1,+1\}\)
        \task \(\{-2,+2\}\)
        \task \(\mathbb{R}\)
        \task \(\oslash\)
    \end{tasks}
\end{question}

\begin{question}%[codigo:EAM201907AMX; concurso:EAM; ano:2019; assunto:; alternativa:]
Os lados de um triângulo medem 30cm, 70cm e 80cm. Ao traçarmos a altura desse triângulo em relação ao maior lado, dividiremos esse lado em dois segmentos. Sendo assim, calcule o valor do menor segmento em centímetros e assinale a opção correta.
    \begin{tasks}
        \task 15.
        \task 14.
        \task 13.
        \task 12.
        \task 11.
    \end{tasks}
\end{question}

\begin{question}%[codigo:EAM201908AMX; concurso:EAM; ano:2019; assunto:; alternativa:]
Um produto custa à vista R\$ 100,00 e pode ser vendido também em 2 parcelas, sendo a primeira no ato da compra, com valor de R\$ 50,00, e a segunda, a vencer em 30 dias, com o valor de R\$ 60,00. Sendo assim, calcule a taxa mensal de juros cobrado pelo vendedor e assinale a opção correta.
    \begin{tasks}
        \task 20\%.
        \task 10\%.
        \task 8\%.
        \task 6\%.
        \task 5\%.
    \end{tasks}
\end{question}

\begin{question}%[codigo:EAM201909AMX; concurso:EAM; ano:2019; assunto:; alternativa:]
Considere o triângulo \(ABC\), isósceles, de lados \(AB=AC\). Seja o ponto \(D\), sobre o lado \(BC\), de forma que o ângulo \(BAD\) é 30\(^\circ\). Seja \(E\) o ponto sobre o lado \(AC\), tal que o ângulo \(EDC\) vale \(x\) graus. Tendo em vista que o segmento \(AD\) e \(AE\) têm as mesmas medidas, é correto afirmar que o valor da quarta parte de \(x\) é:
    \begin{tasks}
        \task 3\(^\circ\).
        \task 3\(^\circ\) 20'.
        \task 3\(^\circ\) 30'.
        \task 3\(^\circ\) 35'.
        \task 3\(^\circ\) 45'.
    \end{tasks}
\end{question}

\begin{question}%[codigo:EAM201910AMX; concurso:EAM; ano:2019; assunto:; alternativa:]
Considere o gráfico abaixo de um função real, definida por \(y= ax + b\):

INSERIR FIGURA

Com base nesse gráfico, é correto afirmar que a equação que define essa função é:
    \begin{tasks}
        \task \(4y = -4x + 16\).
        \task \(4y = -4x + 8\).
        \task \(y = -2x + 4\).
        \task \(y= 2x + 2\).
        \task \(2y = x-2\).
    \end{tasks}
\end{question}

\begin{question}%[codigo:EAM201911AMX; concurso:EAM; ano:2019; assunto:; alternativa:]
Calcule o valor de \(x\), na equação: 
\(\begin{bmatrix}
x & 1  & 1\\
3 & 1 & 1\\
1 & -3 & 1
\end{bmatrix}
=24 \) e assinale a opção correta.
    \begin{tasks}
        \task 11.
        \task 10.
        \task 9.
        \task 8.
        \task 7.
    \end{tasks}
\end{question}

\begin{question}%[codigo:EAM201912AMX; concurso:EAM; ano:2019; assunto:; alternativa:]
Sejam os conjuntos \(A=\{x \in \mathbb{R}; 1 \leq x \leq 4\}, B = \{ y \in \mathbb{R}; 3 \leq y \leq 7\}\). Considerando o conjunto \(A \times B\) (\(A\) cartesiano \(B\)), pode-se afirmar que a diagonal do polígono formado por esse conjunto é representada numericamente por:
    \begin{tasks}
        \task 2.
        \task 3.
        \task 4.
        \task 5.
        \task 6.
    \end{tasks}
\end{question}

\begin{question}%[codigo:EAM201913AMX; concurso:EAM; ano:2019; assunto:; alternativa:]
Seja \(A\) um conjunto com \(n\) elementos, tal que \(n \geq 3\). O número de subconjuntos de \(A\) com dois ou três elementos que podemos construir é igual a:
    \begin{tasks}
        \task \(\frac{(n^2-1)}{6}\).
        \task \(\frac{n-1}{6}\).
        \task \(\frac{n(n^2+1)}{6}\).
        \task \(\frac{n(n^2-1)}{6}\).
        \task \(\frac{n(n^2-1)}{5}\).
    \end{tasks}
\end{question}

\begin{question}%[codigo:EAM201914AMX; concurso:EAM; ano:2019; assunto:; alternativa:]
Observe a figura abaixo.

INSERIR FIGURA

Considerando que os triângulos \(BDA\) e \(BCA\) apresentados acima são, respectivamente, retângulos em \(D\) e \(C\), calcule o valor de \(x\) em função do lado \(c\) e assinale a opção correta.
    \begin{tasks}
        \task \(\sqrt{c^3 - 2}\).
        \task \(\sqrt{c^2 - 1}\). 
        \task \(\sqrt{c^2 + 5}\).
        \task \(\sqrt{c - 3}\).
        \task \(\sqrt{c^2 -3}\).
    \end{tasks}
\end{question}

\begin{question}%[codigo:EAM201915AMX; concurso:EAM; ano:2019; assunto:; alternativa:]
Considerando os conjuntos \(\mathbb{N},\mathbb{Z},\mathbb{Q}\) e \(\mathbb{R}\), coloque V(verdadeiro) ou F(falso) nas sentenças abaixo, assinalando a seguir a opção correta.

\begin{enumerate}[label=(~~)]
    \item \( (\mathbb{N}^{*} \cap \mathbb{Q}) = \mathbb{N}^{*}\).
    \item \( (\mathbb{Z} - \mathbb{Z}) = \mathbb{Z_{+}}\).
    \item \( ( \mathbb{R} \cup \mathbb{Z}) = \mathbb{Q}\).
\end{enumerate}
    \begin{tasks}
        \task (V)(V)(V).
        \task (V)(V)(F).
        \task (V)(F)(F).
        \task (F)(V)(F).
        \task (F)(F)(V).
    \end{tasks}
\end{question}

% \begin{question}%[concurso:EAM; ano:2020; assunto:; alternativa:]
Observe a figura a seguir.

INSERIR FIGURA

Nesta figura, tem-se \(\overline{AB} = \overline{AC} = 9, \overline{BC} = \overline{BD} = 6\) e ângulos \(C\hat{B}Q = Q\hat{B}D\). É correto afirmar que o cosseno do ângulo \(C\hat{B}Q\) é igual a:
    \begin{tasks}
        \task \(\frac{\sqrt{2}}{4}\).
        \task \(\sqrt{2}\).
        \task \(\frac{3}{2}\).
        \task \(\frac{\sqrt{4}}{2}\).
        \task \(\frac{2\sqrt{2}}{3}\).
    \end{tasks}
\end{question}

\begin{question}%[concurso:EAM; ano:2020; assunto:; alternativa:]
Um bar possui um alvo, como o da figura abaixo, para entretenimento dos seus clientes em lançamento de dardos. Esse alvo é formado por figuras combinadas: um semicírculo com diâmetro \(AB\), um semicírculo com diâmetro \(BC\) e um triângulo retângulo \(ABC\), conforme se observa na figura.

INSERIR FIGURA

Se o cateto \(AC\) mede 6dm, a hipotenusa \(AB\) mede 10dm e um cliente de costas para o alvo arremessa um dardo e o acerta, é correto afirmar que a probabilidade de que o dardo tenha acertado a parte sombreada do alvo é dada por uma porcentagem entre:
    \begin{tasks}
        \task 5\% e 15\%.
        \task 15\% e 25\%.
        \task 25\% e 35\%.
        \task 35\% e 45\%.
        \task 45\% e 55\%.
    \end{tasks}
\end{question}

\begin{question}%[concurso:EAM; ano:2020; assunto:; alternativa:]
Para compor a tripulação de um voo, certa companhia de aviação dispõe de 5 pilotos, 3 copilotos, 4 comissários e 6 aeromoças. De quantos modos ela pode escalar uma equipe para um voo, sabendo que esse voo precisa de um piloto, um copiloto, dois comissários e 3 aeromoças?
    \begin{tasks}
        \task 2 140.
        \task 1 920.
        \task 1 800.
        \task 1 750.
        \task 1 280.
    \end{tasks}
\end{question}

\begin{question}%[concurso:EAM; ano:2020; assunto:; alternativa:]
Considere as matrizes \(A\) e \(B\) a seguir:
\[ A = \begin{bmatrix}
        x & 1 \\
        -2 & x 
    \end{bmatrix}
\textrm{ e } 
   B = \begin{bmatrix}
        1 & x \\
        1 & -4 
    \end{bmatrix}
 \]

Existem dois valores \(x_1\) e \( x_2 ~ (x_1 > x_2)\) tal que \(det(A) + det(B) = 0\). É correto afirmar que a expressão \(5x_1 - 3x_2\) é igual a:
    \begin{tasks}
        \task 18.
        \task 13.
        \task 10.
        \task 7.
        \task 6.
    \end{tasks}
\end{question}

\begin{question}%[concurso:EAM; ano:2020; assunto:; alternativa:]
Observe o triângulo a seguir.

INSERIR IMAGEM

No triângulo \(ABC\) traçamos o segmento \(AD\) de forma que \(DC=AC\). Se o ângulo \(B\hat{A}C\) supera em 40\(^\circ\) o ângulo \(ABC\), é correto afirmar que o ângulo \(B\hat{A}D\) mede, em graus:
    \begin{tasks}
        \task 35\(^\circ\).
        \task 30\(^\circ\).
        \task 25\(^\circ\).
        \task 20\(^\circ\).
        \task 15\(^\circ\).
    \end{tasks}
\end{question}

\begin{question}%[concurso:EAM; ano:2020; assunto:; alternativa:]
Para determinar se uma solução é básica, neutra ou ácida calcula-se o potencial hidrogeniônico (Ph) da solução através da fórmula \(PH = -\log [H^{+}]\) onde \(H^{+}\) é a concentração hidrogeniônica da solução. Considere o suco de magnésio com \(H^{+} = 10^{-10}\) e a bile segregada pelo fígado humano com \(H^{+} = 10^{-8}\) e solução classificada por meio dos seguintes parâmetros:

TABELA

Com base nessas informações, é correto afirmar que:
    \begin{tasks}
        \task a bile é básica e o suco de magnésio é ácido.
        \task a bile é ácida e o suco de magnésio é básico.
        \task a bile é básica e o suco de magnésio é básico.
        \task a bile é ácida e o suco de magnésio é ácido.
        \task ambas as soluções são neutras.
    \end{tasks}
\end{question}

\begin{question}%[concurso:EAM; ano:2020; assunto:; alternativa:]
Em um quadrilátero, os ângulos internos são expressos em graus por \(3x + 80, 40 - 3x, 90-5x\) e \(2x + 120\). É correto afirmar que o menor ângulo mede:
    \begin{tasks}
        \task 40\(^\circ\).
        \task 50\(^\circ\).
        \task 60\(^\circ\).
        \task 70\(^\circ\).
        \task 80\(^\circ\).
    \end{tasks}
\end{question}

\begin{question}%[concurso:EAM; ano:2020; assunto:; alternativa:]
Num paralelogramo dois de seus lados adjacentes formam o ângulo de 30\(^\circ\) e medem 5cm e \(5\sqrt{3}\)cm respectivamente. Calcule a diferença entre a diagonal maior e a diagonal menor desse paralelogramo e assinale a opção que apresenta essa diferença.
    \begin{tasks}
        \task \(5(\sqrt{7} - 1)\).
        \task \(5(\sqrt{7} - 2)\).
        \task \(5(\sqrt{3} - 1)\).
        \task \(5\sqrt{3}\).
        \task \(5\sqrt{7}\).
    \end{tasks}
\end{question}

\begin{question}%[concurso:EAM; ano:2020; assunto:; alternativa:]
As raízes do polinômio \(p(x) = x^3 -10x^2 + 29x - 20\) são as dimensões de um paralelepípedo retângulo. É correto afirmar que a área de todas as faces da figura em unidades de área é igual a:
    \begin{tasks}
        \task 28.
        \task 29.
        \task 36.
        \task 48.
        \task 58.
    \end{tasks}
\end{question}

\begin{question}%[concurso:EAM; ano:2020; assunto:; alternativa:]
Um estimativa de dados indica que, caso o preço do ingresso para um jogo de futebol, custe R\$ 20,00, haverá um público de 3 600 pagantes, arrecadando um total de R\$ 72 000,00. Entretanto foi estimado também que, a cada aumento de R\$ 5,00 no preço do ingresso, o público diminuiria em 100 pagantes. Considerando tais estimativas, para que a arrecadação seja a maior possível, o preço unitário do ingresso de tal jogo deve ser:
    \begin{tasks}
        \task R\$ 30,00.
        \task R\$ 60,00.
        \task R\$ 80,00.
        \task R\$ 100,00.
        \task R\$ 120,00.
    \end{tasks}
\end{question}

\begin{question}%[concurso:EAM; ano:2020; assunto:; alternativa:]
Ao resolver a equação \(6445^2 + 3x = 6446^2\), encontramos para \(x\) um número inteiro tal que a soma dos seus algarismos é igual a:
    \begin{tasks}
        \task 14
        \task 18
        \task 22
        \task 26
        \task 28
    \end{tasks}
\end{question}

\begin{question}%[concurso:EAM; ano:2020; assunto:; alternativa:]
No almoxarifado de uma escola, encontram-se numa caixa 60 lápis e 40 canetas, sendo que 24 lápis e 16 canetas são intocados. Ao escolhermos uma peça ao acaso, é correto afirmar que a probabilidade de ser um lápis ou ser um objeto intocado é igual a:
    \begin{tasks}
        \task 84\%.
        \task 76\%.
        \task 60\%.
        \task 50\%.
        \task 36\%.
    \end{tasks}
\end{question}

\begin{question}%[concurso:EAM; ano:2020; assunto:; alternativa:]
Para construir uma ponte entre duas margens de um rio foram marcados, primeiramente, dois pontos \(A\) e \(B\) numa mesma margem distantes 100m e um ponto \(C\) na margem oposta. Utilizando um teodolito (aparelho utilizado para medição de ângulo) descobriram-se as seguintes informações: ângulo \(C\hat{A}B = 30^\circ\) e ângulo \(A\hat{B}C = 75^\circ\). Sabe-se que a ponte deverá ter o menor tamanho possível saindo do ponto \(C\) e chegando a margem oposta. Sendo assim, é correto afirmar que o comprimento dessa ponte será igual a:
    \begin{tasks}
        \task 20m.
        \task 30m.
        \task 40m.
        \task 50m.
        \task 60m.
    \end{tasks}
\end{question}

\begin{question}%[concurso:EAM; ano:2020; assunto:; alternativa:]
Na figura abaixo tem-se um pentágono regular \(ABCDE\) no qual devem ser traçadas as diagonais \(CE\) e \(BD\) e um segmento \(AM\), onde \(M\) é o ponto médio do lado \(CD\). Sabe-se também que \(AM\) passa pelo ponto de intersecção das diagonais traçadas.

INSERIR IMAGEM

Com base nessas informações, é correto afirmar que o número \(n\) de triângulos na figura formada, após os traços, é tal que \(n\) vale:
    \begin{tasks}
        \task 6.
        \task 7.
        \task 8.
        \task 9.
        \task 10.
    \end{tasks}
\end{question}

\begin{question}%[concurso:EAM; ano:2020; assunto:; alternativa:]
Considere a coroa circular formada pelas circunferências \(L_1\) e \(L_2\) cuja soma dos raios vale 0,4dm, conforme figura a seguir.

INSERIR IMAGEM

Se a área da coroa é igual a \(\pi\)dm\(^\circ\), é correto afirmar que a diferença positiva em dm entre os comprimento das circunferências \(L_1\) e \(L_2\) é igual a:
    \begin{tasks}
        \task \(2\pi\).
        \task \(3\pi\).
        \task \(4\pi\).
        \task \(5\pi\).
        \task \(6\pi\).
    \end{tasks}
\end{question}

% \begin{question}%[concurso:EAM; ano:2021; assunto:; alternativa:]
Dadas as progressões aritméticas \(A: (2,x,8)\), \(B:(5,y,11)\) e \(C:(8,z,14)\). Determine a soma dos seis primeiros termos da \(PA(x,y,z, \ldots)\) e marque a opção correta.
    \begin{tasks}
        \task 15.
        \task 24.
        \task 33.
        \task 65.
        \task 75.
    \end{tasks}
\end{question}

\begin{question}%[concurso:EAM; ano:2021; assunto:; alternativa:]
Dada a equação \(\frac{p^q - p^{-q}}{p^q + p^{-q}} = r\), onde \(q \in \mathbb{R}\) e \(0 < p \neq 1\), o valor de \(p^{2q}\) é:
    \begin{tasks}
        \task \(\frac{1-r}{r+1}\).
        \task \(r\).
        \task \(\frac{r+1}{1-r}\).
        \task \(r+1\).
        \task \(r-1\).
    \end{tasks}
\end{question}

\begin{question}%[concurso:EAM; ano:2021; assunto:; alternativa:]
Determine o cosseno de \(1935^\circ\) e marque a opção correta.
    \begin{tasks}
        \task \(\frac{\sqrt{2}}{2}\).
        \task \(1\).
        \task \(\frac{1}{2}\).
        \task \(-\frac{1}{2}\).
        \task \(\frac{- \sqrt{2}}{2}\).
    \end{tasks}
\end{question}

\begin{question}%[concurso:EAM; ano:2021; assunto:; alternativa:]
Encontre a medida do segmento \(\overline{CD}\) na figura abaixo, sabendo que \(BCDE\) é um retângulo e \(\overline{BA} = 75\)cm, e marque a opção correta.

INSERIR IMAGEM
    \begin{tasks}
        \task 25 cm.
        \task 25\(\sqrt{3}\) cm.
        \task 50 cm.
        \task 75 cm.
        \task 75 \(\sqrt{3}\) cm.
    \end{tasks}
\end{question}

\begin{question}%[concurso:EAM; ano:2021; assunto:; alternativa:]
Encontre o valor de \(K\) para que o resto da divisão de \(P(x) = 5x^2 - 4kx + 2\) por \(2x-6\) seja \(5\), e marque a opção correta.
    \begin{tasks}
        \task \(\frac{9}{2}\).
        \task \(\frac{7}{2}\).
        \task \(\frac{11}{2}\).
        \task \(\frac{10}{2}\).
        \task \(\frac{12}{2}\).
    \end{tasks}
\end{question}

\begin{question}%[concurso:EAM; ano:2021; assunto:; alternativa:]
Determine o valor de \(\log_{3\sqrt{3}} 27\) e marque a opção correta.
    \begin{tasks}
        \task 5.
        \task 4.
        \task 3.
        \task 2.
        \task 1.
    \end{tasks}
\end{question}

\begin{question}%[concurso:EAM; ano:2021; assunto:; alternativa:]
A soma dos ângulos internos do polígono que possui o número de lados igual ao número de diagonais é:
    \begin{tasks}
        \task  90\(^\circ\).
        \task 180\(^\circ\).
        \task 540\(^\circ\).
        \task 560\(^\circ\).
        \task 720\(^\circ\).
    \end{tasks}
\end{question}

\begin{question}%[concurso:EAM; ano:2021; assunto:; alternativa:]
Assinale a opção que contém o número de anagramas da palavra APRENDIZ.
    \begin{tasks}
        \task 40 300.
        \task 40 320.
        \task 40 330.
        \task 40 340.
        \task 40 350.
    \end{tasks}
\end{question}

\begin{question}%[concurso:EAM; ano:2021; assunto:; alternativa:]
Em uma loja de eletroeletrônicos, um aparelho de R\$ 1 450,00 na virada do mês, passou a custar R\$ 1 740,00. O preço desse aparelho teve um aumento de:
    \begin{tasks}
        \task 20\%.
        \task 25\%.
        \task 30\%.
        \task 35\%.
        \task 40\%.
    \end{tasks}
\end{question}

\begin{question}%[concurso:EAM; ano:2021; assunto:função exponencial; alternativa:]
Em uma cidade, a população têm sido contaminada pelo novo Sars-coV-2. Suponha que o número de contaminados pelo vírus seja dado pela função \(f(x) = \left( 10 - \frac{1}{2^x}\right) \cdot 10 000\), onde \(x\) representa a quantidade de meses. Assinale a opção que apresenta o número de contaminados, nessa cidade, no terceiro mês.
    \begin{tasks}
        \task 98 000.
        \task 98 700.
        \task 98 720.
        \task 98 750.
        \task 98 950.
    \end{tasks}
\end{question}

\begin{question}%[concurso:EAM; ano:2021; assunto:; alternativa:]
Se as matrizes \(A = (a_{ij}), B = (b_{ij})\) e \(C = (c_{ij})\), ambas quadradas e de 3\textordfeminine{} ordem, estão definidas:

\(A=\begin{cases}
    i^j, \textrm{ se } i> j\\
    i + j, \textrm{ se } i = j\\
    -i, \textrm{ se } i < j
\end{cases} , B = b_{ij = i^2}\) e \(C = A + B\). Nesse caso, o cofator de \(C_{32}\) é:
    \begin{tasks}
        \task -18.
        \task -6.
        \task -1.
        \task 6.
        \task 18.
    \end{tasks}
\end{question}

\begin{question}%[concurso:EAM; ano:2021; assunto:; alternativa:]
Dada uma função exponencial \(f(x) = a^x\), a respeito de suas características é correto afirmar que a função é:
    \begin{tasks}
        \task decrescente para a base \(a\) maior que \(1 (a>1)\).
        \task crescente para \(x\) maior que \(0\).
        \task crescente se a base \(a\) for igual a \(1(a=1)\).
        \task crescente para \(x\) maior que \(0\) e menor \(1(0<x<1)\).
        \task decrescente para a base \(a\) maior que \(0\) e menor que \(1(0<a<1)\).
    \end{tasks}
\end{question}

\begin{question}%[concurso:EAM; ano:2021; assunto:; alternativa:]
Para qualquer \(a\) real, a expressão: \(4^a + 4^{a+1} + (4^a \cdot 16) + 4^{a+3} + 4^a \cdot 256 + 4^{a \div 5}\) é equivalente a:
    \begin{tasks}
        \task \( 4^{6a} + 15\).
        \task \(4^a + 15\).
        \task \(1365^a\).
        \task \(1365 \cdot 4^a\).
        \task \(1365^{2a}\).
    \end{tasks}
\end{question}

\begin{question}%[concurso:EAM; ano:2021; assunto:; alternativa:]
Uma pesquisa de mercado sobre o consumo de três marcas de café \(A,B\) e \(C\), apresentou os seguintes resultados:

\begin{itemize}
    \item 60\% consomem o produto \(A\);
    \item 51\% consomem o produto \(B\);
    \item 15\% consomem o produto \(C\);
    \item 5\% consomem os três produtos;
    \item 11\% consomem os produtos \(A\) e \(B\); e
    \item 10\% consomem os produtos \(B\) e \(C\).
\end{itemize}
Qual é o percentual relativo à quantidade de pessoas que consomem, simultaneamente, os produtos \(A\) e \(C\) sem consumir o \(B\)?
    \begin{tasks}
        \task 3\%.
        \task 5\%.
        \task 7\%.
        \task 9\%.
        \task 11\%.
    \end{tasks}
\end{question}

\begin{question}%[concurso:EAM; ano:2021; assunto:; alternativa:]
Determine a área hachurada, no gráfico abaixo, sabendo que \(V\) é o vértice da parábola, e marque a opção correta.

INSERIR IMAGEM
    \begin{tasks}
        \task 40.
        \task 50.
        \task 60.
        \task 70.
        \task 80.
    \end{tasks}
\end{question}

% \begin{question}%[concurso:EAM; ano:2022; assunto:; alternativa:]
Observe a figura abaixo:
INSERIR FIGURA
Se \(ABCD\) é um quadrado e \(ABP\) um triângulo equilátero, determine o ângulo \(x\) e assinale a opção correta.
    \begin{tasks}
        \task \(135^\circ\).
        \task \(105^\circ\).
        \task \(100^\circ\).
        \task \(97^\circ\).
        \task \(95^\circ\).
    \end{tasks}
\end{question}

\begin{question}%[concurso:EAM; ano:2022; assunto:probabilidade; alternativa:]
Uma das sensações nos jogos online é o \textit{Call of Duty - WARZONE}, pois, em um dos seus modos de jogo a equipe vencedora é a última que sobrevive. Considera um jogador do \textit{WARZONE} chamado NEGUEBA. Suponto que em uma partido online no \textit{WARZONE} existam sempre 4 caminhos para tentar derrubar um oponente, sendo que em apenas um deles é possível derrubar. Assim, para cada caminho, NEGUEBA tem probabilidade de \(\frac{1}{4}\) de escolher o o que vai derrubar um oponente se ele está adivinhando e 1 se ele sabe esse caminho. NEGUEBA sabe 10\% dos caminhos para derrubar um oponente. Se ele derrubou um dos oponentes, qual é a probabilidade dele ter adivinhado o caminho?
    \begin{tasks}
        \task \(\frac{9}{13}\).
        \task \(\frac{4}{5}\).
        \task \(\frac{8}{13}\).
        \task \(\frac{7}{16}\).
        \task \(\frac{3}{7}\).
    \end{tasks}
\end{question}

\begin{question}%[concurso:EAM; ano:2022; assunto:; alternativa:]
Determine a equação reduzida da elipse cujo eixo menor tem por extremos os focos da hipérbole \(x^2 - y^2 = -1\) e cuja excentricidade é o inverso da excentricidade da hipérbole dada, como mostra a figura abaixo, e assinale a opção correta.
INSERIR IMAGEM
    \begin{tasks}
        \task \(\frac{x^2}{4} + \frac{y^2}{2} = 1 \).
        \task \(\frac{x^2}{3} + \frac{y^2}{2} = 1 \).
        \task \(\frac{x^2}{2} + \frac{y^2}{4} = 1 \).
        \task \(\frac{x^2}{2} + \frac{y^2}{3} = 1 \).
        \task \(x^2 + y^2 = 1\).
    \end{tasks}
\end{question}

\begin{question}%[concurso:EAM; ano:2022; assunto:; alternativa:]
Assinale a opção que apresenta a soma de todos os inteiros que divididos por 11 dão resto 7 e estão compreendidos entre 200 e 400.
    \begin{tasks}
        \task 5373.
        \task 5431.
        \task 5578.
        \task 5691.
        \task 5743.
    \end{tasks}
\end{question}

\begin{question}%[concurso:EAM; ano:2022; assunto:; alternativa:]
As arestas laterais de uma pirâmide medem 52cm e sua base é um triângulo isósceles cujos lados medem 24cm, \(12\sqrt{10}\)cm e \(12\sqrt{10}\)cm. Sabendo que a projeção do vértice da pirâmide na base triangular é o centro de sua circunferência circunscrita, determine a altura dessa pirâmide e assinale a opção correta.
    \begin{tasks}
        \task 12cm.
        \task 16cm.
        \task 30cm.
        \task 36cm.
        \task 48cm.
    \end{tasks}
\end{question}

\begin{question}%[concurso:EAM; ano:2022; assunto:; alternativa:]
Considere a elipse \(E\) com centro na origem, um dos focos em \(F_1 \left( 0, \sqrt{\frac{2}{3}} \right)\) e que passa pelo ponto \(P \left( \frac{1}{2}, \frac{1}{2} \right)\), como mostrado na figura abaixo. Assinale a opção correta que apresenta a excentricidade de \(E\).
INSERIR IMAGEM
    \begin{tasks}
        \task \(\frac{1}{6}\).
        \task \(\frac{1}{2}\).
        \task \(\sqrt{\frac{2}{3}}\).
        \task \(1\).
        \task \(\sqrt{\frac{3}{2}}\).
    \end{tasks}
\end{question}

\begin{question}%[concurso:EAM; ano:2022; assunto:; alternativa:]
Um nutricionista deseja preparar uma refeição diária equilibrada em vitaminas \(A,B\) e \(C\). Para isso ele dispõe de 3 tipos de alimentos \(X,Y\) e \(Z\). O alimento \(X\) possui uma unidade de vitamina \(A\), 10 unidades de vitamina \(B\) e uma unidade de vitamina \(C\). O alimento \(Y\) possui 9 unidades de vitamina \(A\), uma de vitamina \(B\) e uma unidade de vitamina \(C\). O alimento \(Z\) possui 2 unidades de vitamina \(A\), 2 unidades de vitamina \(B\) e 2 unidades de vitamina \(C\). Sabendo que para uma alimentação diária equilibrada em vitamina deve conter 160 unidade de vitamina \(A\), 170 unidades de vitamina \(B\) e 140 unidades de vitamina \(C\), calcule a soma das quantidades de alimentos que deverão ser utilizadas na refeição e assinale a opção correta.
    \begin{tasks}
        \task 45.
        \task 50.
        \task 55.
        \task 60.
        \task 65.
    \end{tasks}
\end{question}

\begin{question}%[concurso:EAM; ano:2022; assunto:; alternativa:]
Encontre os valores dos arcos \(x\) e \(y\) indicados na figura abaixo e assinale a opção correta.
INSERIR IMAGEM
    \begin{tasks}
        \task \(x = 30^\circ\) e \(y = 90^\circ\).
        \task \(x = 45^\circ\) e \(y = 90^\circ\).
        \task \(x = 45^\circ\) e \(y = 75^\circ\).
        \task \(x = 60^\circ\) e \(y = 75^\circ\).
        \task \(x = 90^\circ\) e \(y = 60^\circ\).
    \end{tasks}
\end{question}

\begin{question}%[concurso:EAM; ano:2022; assunto:; alternativa:]
Uma esfera com centro em \(O\) possui volume igual a \(\frac{1372\pi}{3}\) cm\(^2\). Se tomarmos um plano e o fizermos interceptar essa esfera a uma distância \(d\) do seu centro, a seção plana circular resultante, de centro \(O'\), terá área igual a \(24\pi\) cm\(^2\) (figura abaixo). Assim, de acordo com os dados, calcule o valor de \(d\), ou seja \(\overline{OO'}\), e assinale a opção correta.

INSERIR IMAGEM
    \begin{tasks}
        \task 1cm.
        \task 3cm.
        \task 5cm.
        \task 7cm.
        \task 10cm.
    \end{tasks}
\end{question}

\begin{question}%[concurso:EAM; ano:2022; assunto:; alternativa:]
Considere duas fontes de luz, \(A\) e \(B\), situadas no eixo das abcissas, com \(A\) na origem. A fonte \(B\) é 4 vezes mais brilhante do que a fonte \(A\) e distam 15m entre si. Suponha que um objeto \(C\) é posto no eixo das abcissas entre \(A\) e \(B\). Sabendo que a luminosidade em \(C\) é diretamente proporcional à intensidade da fonte e inversamente proporcional ao quadrado da distância desse ponto à mesma fonte. A que distância de \(A\) deve estar \(C\) para que seja iluminado igualmente por ambas as fontes?
    \begin{tasks}
        \task 1m.
        \task 3m.
        \task 5m.
        \task 6m.
        \task 7m.
    \end{tasks}
\end{question}

\begin{question}%[concurso:EAM; ano:2022; assunto:; alternativa:]
Assinale a opção que apresenta o valor de \(x\) para o qual é solução da equação \(\log_9 x + \log_{27} x - \log_3 x = -1\).
    \begin{tasks}
        \task 603.
        \task 729.
        \task 831.
        \task 867.
        \task 906.
    \end{tasks}
\end{question}

\begin{question}%[concurso:EAM; ano:2022; assunto:; alternativa:]
Calcule a área \(S\) e o perímetro \(P\) do triângulo \(ABA'\) abaixo e assinale a opção correta.

INSERIR IMAGEM
    \begin{tasks}
        \task \(S = \sqrt{2}\) e \(P=1+ \sqrt{3}\).
        \task \(S = \sqrt{3}\) e \(P=5 + \sqrt{2}\).
        \task \(S = 5\sqrt{2}\) e \(P=\sqrt{3}\).
        \task \(S = 8\sqrt{3}\) e \(P=4(3+ \sqrt{3})\).
        \task \(S = 10\sqrt{3}\) e \(P=2(2+\sqrt{3})\).
    \end{tasks}
\end{question}

\begin{question}%[concurso:EAM; ano:2022; assunto:; alternativa:]
Sabendo que a reta \(r\) é determinada pelos pontos de interseção da função \(f(x)= x^2-x\) com a sua inversa \(f^{-1}(x)\), como representado na figura abaixo, e seja o menor segmento de reta \(PP'\) que une o ponto \(P(10,0)\) a esta reta, com \(P'\in r\). Considere o triângulo retângulo \(OP'P\) sendo \(O\) a origem do eixo cartesiano e reto em \(P'\). Desse modo, encontre o tamanho do segmento \(PP'\) e assinale a opção correta.

INSERIR IMAGEM
    \begin{tasks}
        \task \(\sqrt{2}\).
        \task \(\sqrt{3}\).
        \task \(2\sqrt{3}\).
        \task \(5\sqrt{2}\).
        \task \(5\sqrt{3}\).
    \end{tasks}
\end{question}

\begin{question}%[concurso:EAM; ano:2022; assunto:; alternativa:]
Um vídeo game é vendido à vista por R\$ 2 000,00 ou a prazo com R\$ 400,00 de entrada e mais uma parcela de R\$ 1 800,00 quatro meses após a compra. Assinale a opção que apresenta a taxa mensal de juros compostos do financiamento. Considere apenas 3 casas decimais e sem arredondamento.
    \begin{tasks}
        \task 2,3\%.
        \task 2,9\%.
        \task 3,3\%.
        \task 4,0\%.
        \task 4,4\%.
    \end{tasks}
\end{question}

\begin{question}%[concurso:EAM; ano:2022; assunto:; alternativa:]
Sabe-se que \((1 - \cos^2(x))(\cotg^2(x) + 1) = A\) para \(x\) diferente de \(k\pi\), com \(k \in \mathbb{Z}\), e que \(\frac{\sec^(x) - 1}{\tg^2(x) + 1} = B\), quando \(\sen(x) = \frac{\sqrt{2}}{2}\). Assim, assinale a opção que apresenta o valor de \(B^A\).
    \begin{tasks}
        \task \(0\).
        \task \(\frac{1}{2}\).
        \task \(1\).
        \task \(\frac{3}{2}\).
        \task \(2\).
    \end{tasks}
\end{question}

% \begin{question}%[codigo:EAM202301AMX; concurso:EAM; ano:2023; assunto:; alternativa:E]
O chão da Sala de Estado da EAMCE é retangular e suas dimensões são 3,52m e 4,16m. Esse chão será revestido com pisos quadrados, de dimensões iguais, inteiros, de forma que não fiquem espaços vazios entre pisos vizinhos. Os pisos serão escolhidos de modo que tenham a maior dimensão possível. Com base nessa situação, assinale a opção que apresenta o intervalo que contém a medida do lado do piso ideal.
    \begin{tasks}
        \task Menos de 15 cm.
        \task Mais de 15 cm e menos de 20 cm.
        \task Mais de 20 cm e menos de 25 cm.
        \task Mais de 25 cm e menos de 30 cm.
        \task Mais de 30 cm.
    \end{tasks}
\end{question}

\begin{question}%[codigo:EAM202302AMX; concurso:EAM; ano:2023; assunto:; alternativa:B]
Migulito e Ditão são dois integrantes de um seleto grupo de vinte tricolores. Será formada uma comissão de cinco pessoas entre seus membros para organizar a feste de confraternização do grupo. Com base nessas informações, assinale a opção que indica de quantas maneiras distintas essa comissão poderá ser formada, de modo que apenas um deles esteja presente.
    \begin{tasks}
        \task 9 180.
        \task 6 120.
        \task 5 400.
        \task 4 590.
        \task 3 060.
    \end{tasks}
\end{question}

\begin{question}%[codigo:EAM202303AMX; concurso:EAM; ano:2023; assunto:; alternativa:E]
Em um exercício da Marinha do Brasil, cinco navios estavam posicionados nos vértices de um pentágono regular imaginário. Assinale a opção que indica a maior distância entre dois desses navios, sabendo que a menor distância entre dois navios mais próximos é 100 milhas marítimas. Dados: \(\sqrt{2} = 1,41\) ; \( \sqrt{3} = 1,73\) e \(\sqrt{5} = 2,24\).
    \begin{tasks}
        \task 122 milhas marítimas.
        \task 132 milhas marítimas.
        \task 142 milhas marítimas.
        \task 152 milhas marítimas.
        \task 162 milhas marítimas.
    \end{tasks}
\end{question}

\begin{question}%[codigo:EAM202304AMX; concurso:EAM; ano:2023; assunto:; alternativa:C]
Durante um exercício naval, a Fragata Constituição lançou um míssil antinavio de superfície (MANSUP), cuja trajetória foi determinada pela parábola de equação cartesiana \(y=-x^2 + 20x\), na qual \(y\) representa a altura do míssil e \(x\), o tempo ocorrido após o lançamento. Do mesmo ponto de lançamento do MANSUP, outro míssil lançado, a fim de interceptá-lo no ponto mais alto da sua trajetória. Sabendo que a trajetória do segundo míssil foi retilínea, assinale a opção que apresenta a equação cartesiana desa trajetória.
    \begin{tasks}
        \task \(y = x\)
        \task \(y = 5x\)
        \task \(y = 10x\)
        \task \(y = 15x\)
        \task \(y = 20x\)
    \end{tasks}
\end{question}

\begin{question}%[codigo:EAM202305AMX; concurso:EAM; ano:2023; assunto:; alternativa:]
A taxa de crescimento da população de uma colônia de bactérias é de 2\% ao mês. Assinale a opção que indica o intervalo de tempo em que o número de bactérias dessa colônia dobra: Dados: \(\log 0,2 = 0,70\), \(\log 2 = 0,30\), \(\log 1,2 = 0,08\) e \(\log 1,02 = 0,008\).

    \begin{tasks}
        \task Durante o 35\textordmasculine~ mês após o início da observação.
        \task Durante o 36\textordmasculine~ mês após o início da observação.
        \task Durante o 37\textordmasculine~ mês após o início da observação.
        \task Durante o 38\textordmasculine~ mês após o início da observação.
        \task Durante o 39\textordmasculine~ mês após o início da observação.
    \end{tasks}
\end{question}

\begin{question}%[codigo:EAM202306AMX; concurso:EAM; ano:2023; assunto:; alternativa:D]
Pelo ponto médio da diagonal de um cubo de aresta 2 cm foi traçado um plano perpendicular a essa diagonal. Assinale a opção que apresenta a área da figura plana obtida pela interseção desse plano com as faces do cubo.
    \begin{tasks}
        \task \(\sqrt{3}\) cm\(^2\).
        \task \(2\sqrt{3}\) cm\(^2\).
        \task \(3\sqrt{3}\) cm\(^2\).
        \task \(4\sqrt{3}\) cm\(^2\).
        \task \(5\sqrt{3}\) cm\(^2\).
    \end{tasks}
\end{question}

\begin{question}%[codigo:EAM202307AMX; concurso:EAM; ano:2023; assunto:; alternativa:]
Um quadrilátero de vértices consecutivos \(A_1,A_2,A_3,A_4\), tem as distâncias entre os vértices \(A_1\) e \(A_3\) indicadas na matriz

\[
A = \begin{bmatrix}
0   & 4,5 & x & 4,5\\
4,5 & 0   & 3 & 3  \\
y   & 3   & 0 & 3  \\
4,5 & 3   & 3 & 0
\end{bmatrix}
\]

pelo elemento \(a_{ij}\), com valores em \(cm\) e \(x,y \in \mathbb{R}\). Determine a área desse quadrilátero e assinale a opção correta.
    \begin{tasks}
        \task \(\frac{9}{2} (2\sqrt{2} + \sqrt{3})\) cm\(^2\)
        \task \(\frac{9}{4} (2\sqrt{2} + \sqrt{3})\) cm\(^2\)
        \task \(\frac{9}{2} (\sqrt{2} + \sqrt{3})\) cm\(^2\)
        \task \(\frac{9}{4} (\sqrt{2} + \sqrt{3})\) cm\(^2\)
        \task \(\frac{9}{2} (2\sqrt{2} - \sqrt{3})\) cm\(^2\)
    \end{tasks}
\end{question}

\begin{question}%[codigo:EAM202308AMX; concurso:EAM; ano:2023; assunto:; alternativa:B]
Os instrutores Adilson (1), Daverson (2), Estácio (3), Isnard (4) e Vicente (5) foram convocados para elaborar o CPAEAM2023. Eles votaram entre si e elegeram o presidente dessa banca. Seus votos foram organizados segundo a matriz \(P\) abaixo, em que cada elemento \(p_{ij}\) é igual a \(1\)(um), se \(i\) votou em \(i\), e \(0\), se \(i\) não votou em \(i\).

\[
P = \begin{bmatrix}
0 & 1 & 0 & 1 & 0\\
1 & 1 & 0 & 0 & 0\\
0 & 0 & 1 & 0 & 1\\
0 & 0 & 1 & 0 & 0\\
1 & 1 & 0 & 1 & 0
\end{bmatrix}
\]

O número de votos foi livre e cada um deles pode votar em si mesmo. Assinale a opção que apresenta quem foi escolhido como presidente da banca.
    \begin{tasks}
        \task Adilson
        \task Daverson
        \task Estácio
        \task Isnard
        \task Vicente
    \end{tasks}
\end{question}

\begin{question}%[codigo:EAM202309AMX; concurso:EAM; ano:2023; assunto:; alternativa:A]
Na recepção da passagem de comando da EAMSC os drinks foram servidos em taças cônicas de 320 ml. Um dos convidados que estava com a taça completamente cheia, resolveu beber a quantidade do drink suficiente para que a bebida restante ficasse na metade da altura da taça, sem considerar sua base. Com base nessas informações, assinale a opção que apresenta a quantidade de bebida que ele sorveu nesse gole.
    \begin{tasks}
        \task 280 ml.
        \task 200 ml.
        \task 160 ml.
        \task 80 ml.
        \task 40 ml.
    \end{tasks}
\end{question}

\begin{question}%[codigo:EAM202310AMX; concurso:EAM; ano:2023; assunto:; alternativa:C]
Considere as equações \(x^2 - 9y^2 - 6x - 18y - 9 = 0\), \(x^2 + y^2 - 2x + 4y + 1 = 0\) e \(x^2 -4x - 4y + 8 = 0\), com \((x,y) \in \mathbb{R}^2\). Analise e assinale a opção que apresenta, respectivamente, as representações geométricas das equações.
    \begin{tasks}
        \task Hipérbole, elipse, parábola.
        \task Hipérbole, circunferência, reta.
        \task Hipérbole, circunferência, parábola.
        \task Elipse, circunferência, parábola.
        \task Elipse, circunferência, reta.
    \end{tasks}
\end{question}

\begin{question}%[codigo:EAM202311AMX; concurso:EAM; ano:2023; assunto:; alternativa:C]
Sejam \(a\) e \(b\) as soluções reais da equação
\(\frac{4 + \sqrt{2x^2 - 7}}{\sqrt{2x^2-7}} = \sqrt{2x^2 - 7} + 4\), com \(a > b\).
\\
Assinale a opção que apresenta o valor correto para \(a^b\).
    \begin{tasks}
        \task \(-4\).
        \task \(-0,25 \).
        \task \( 0,25\).
        \task \( 1\).
        \task \( 4\).
    \end{tasks}
\end{question}

\begin{question}%[codigo:EAM202312AMX; concurso:EAM; ano:2023; assunto:; alternativa:E]
O polinômio \(P(x) = -x^3 + 2x^2 + mx + n\) é divisível simultaneamente pelos polinômios \(Q(x) = x - 1\) e \(R(x) = x-2\). Determine o valor de \(m-n\) e assinale a opção correta.
    \begin{tasks}
        \task \(-1 \).
        \task \( 0\).
        \task \( 1\).
        \task \( 2\).
        \task \( 3\).
    \end{tasks}
\end{question}

\begin{question}%[codigo:EAM202301AMX; concurso:EAM; ano:2023; assunto:; alternativa:]
Através de um ponto \(P\) qualquer, tomado dentro de um triângulo, são traçadas três retas paralelas aos lados desse triângulo. Essas retas dividem a superfície do triângulo em seis partes, três das quais são triângulos de áreas \(4\) cm\(^2\), \(9\) cm\(^2\) e \(16\) cm\(^2\). Assinale a opção que apresenta a àrea do triângulo original.
    \begin{tasks}
        \task \(64\) cm\(^2\).
        \task \(72\) cm\(^2\).
        \task \(81\) cm\(^2\).
        \task \(90\) cm\(^2\).
        \task \(100\) cm\(^2\).
    \end{tasks}
\end{question}

\begin{question}%[codigo:EAM202301AMX; concurso:EAM; ano:2023; assunto:; alternativa:]
Um concurso para as Escolas de Aprendizes-Marinheiros ofereceu uma certa quantidade de vagas, das quais \(1/5\) foi destinado para a área de Eletroeletrônica. O restante foi dividido igualmente entre as áreas Profissional de Apoio e Mecânica. As áreas de Mecânica e Eletroeletrônica contam com uma subespecialidade em comum, chamada Armamento de Aviação, que recebeu 30\% das vagas de Mecânica e 50\% das vagas de Eletroeletrônica. Em relação ao concurso em questão, determine o percentual de vagas destinadas a Armamento de Aviação e assinale a opção correta.
    \begin{tasks}
        \task 22\%.
        \task 26\%.
        \task 30\%.
        \task 40\%.
        \task 80\%.
    \end{tasks}
\end{question}

\begin{question}%[codigo:EAM202301AMX; concurso:EAM; ano:2023; assunto:; alternativa:]
Ao se tentar abrir uma porta com um chaveiro que contém várias chaves parecidas, há quem afirme que a porta será aberta somente na última tentativa. Pedro recebeu a tarefa de guardar equipamentos no paiol e, para tal, deram-lhe um chaveiro contento oito chaves. Assim, calcule a probabilidade de que ele acerte somente na última tentativa e assinale a opção correta.
    \begin{tasks}
        \task 12,5\%.
        \task 25,0\%.
        \task 50,0\%.
        \task 75,0\%.
        \task 87,5\%.
    \end{tasks}
\end{question}

% \vfill
\end{multicols}
\end{document}
